\documentclass[12pt]{article}
\usepackage[margin=3cm]{geometry}
\usepackage{titlesec} % pacote para formatar títulos de seções
\usepackage{tocloft} % pacote para formatar o sumário
\usepackage{hyperref} % pacote para adicionar links no sumário
\usepackage[portuguese]{babel}
\usepackage{setspace}
\usepackage[dvipsnames]{xcolor}
\usepackage{mdframed}
\usepackage{graphicx}
\usepackage{amsmath}
\usepackage{amsthm}
\usepackage{amssymb}
\usepackage{tikz}
\usepackage[noend]{algpseudocode}
\usepackage{algorithm}
%% Floating package
\usepackage{floatflt,epsfig,epsf}
\renewcommand{\baselinestretch}{1.5}
% Renomeia os rótulos para as estruturas de controle
\floatname{algorithm}{Algoritmo}
\renewcommand{\listalgorithmname}{Lista de Algoritmos}
\renewcommand{\algorithmicrequire}{\textbf{Requer:}}
\renewcommand{\algorithmicensure}{\textbf{Garante:}}
\renewcommand{\algorithmicend}{\textbf{Fim}}
\renewcommand{\algorithmicif}{\textbf{Se}}
\renewcommand{\algorithmicthen}{\textbf{Então}}
\renewcommand{\algorithmicelse}{\textbf{Senão}}
\renewcommand{\algorithmicfor}{\textbf{Para}}
\renewcommand{\algorithmicdo}{\textbf{faça}}
\renewcommand{\algorithmicwhile}{\textbf{Enquanto}}
\renewcommand{\algorithmicfunction}{\textbf{Algoritmo}}

\newcommand{\opt}{\ensuremath{\mathrm{opt}}}
\newcommand{\OPT}{\ensuremath{\mathrm{opt}}}
\newcommand{\Opt}{\ensuremath{\mathrm{opt}}}
\newcommand{\val}{\ensuremath{\mathrm{val}}}

\newtheorem{theorem}{Teorema}[section]
\newtheorem{corollary}{Corolário}[theorem]
\newtheorem{lemma}[theorem]{Lema} 

\newtheorem{definition}[theorem]{Definição}
\newtheorem{problem}[theorem]{Problema}
\newcommand{\scs}{\mbox{{\sc scs}}}
\newcommand{\NP}{\mathit{NP}}
\newcommand{\red}[1]{{\color{red}{#1}}}
% \setlength{\parskip}{0.1cm}
\sloppy
\begin{document}
\begin{center}
  
{\Large {\bf Algoritmos de Aproximação para \\Problemas de Clustering}
}

\vspace{0.2cm}
{\small 
{\bf Orientadora:} Cristina Gomes Fernandes \\
{\bf Aluno:} João Guilherme Alves Santos
}

\vspace{5mm} 

\begin{abstract}
Este é o projeto de pesquisa do aluno de graduação João Guilherme Alves Santos sob supervisão da Profa.\ Dra.\ Cristina Gomes Fernandes. O objetivo desse projeto é estudar e pesquisar algoritmos de aproximação para problemas de clustering. O material estudado fornecerá a João Guilherme o conhecimento necessário para buscar um futuro mestrado na área.
\end{abstract}

\end{center}
\newpage

\tableofcontents

\newpage

\section{Introdução}

Problemas de otimização têm o objetivo de encontrar um ponto ótimo de uma função definida sobre um certo domínio. Especificamente, os problemas de otimização combinatória têm domínio finito. Muitos desses problemas são $\NP$-difíceis. Para problemas $\NP$-difíceis, não existem algoritmos eficientes que encontrem uma solução ótima para toda instância de tais problemas a menos que $P = \NP$.

Nesse contexto, algoritmos de aproximação surgiram. A ideia é abrir mão de encontrar soluções ótimas para encontrar, eficientemente, uma solução cujo valor garante uma relação pré-estabelecida com o valor ótimo. 

Clustering refere-se a uma classe de problemas de otimização cujo objetivo é agrupar objetos de maneira que objetos no mesmo cluster apresentem mais semelhanças quando comparados a objetos em clusters diferentes. Tais semelhanças serão definidas pelo problema em questão. Neste projeto, iremos estudar, sob o ponto de vista de algoritmos de aproximação, três problemas de clustering $\NP$-difíceis: localização de instalações, $k$-mediana e $k$-centros. 

Localização de instalações é um problema que visa determinar a melhor localização para instalações, como fábricas ou depósitos, com base no custo de abertura das instalações e custos de transporte. Além disso, pode ser modelado para outras aplicações como problemas de posicionamento de caches em um computador ou problemas de projeto de redes.

Existem várias versões do problema de localização de instalações, a mais simples delas é a versão sem capacidades em que as instalações não têm limitações para suprir os clientes.

No problema de localização de instalações sem capacidades, temos um grafo $(F,D)$-bipartido completo em que $D$ é o conjunto de clientes a serem atendidos e $F$ o conjunto de instalações que podem ser abertas. Para cada cliente $j \in D$ e cada instalação $i \in F$, há um custo $c_{ij}$ para a aresta $ij$ em associar o cliente $j$ à instalação $i$. Além disso, existe um custo de abertura $f_i$ para cada instalação $i \in F$. O objetivo do problema é escolher um conjunto $F' \subseteq F$ e uma função que associe cada cliente a uma instalação aberta tal que o custo total de abertura das instalações em~$F'$ somado ao custo de associação de cada cliente $j \in D$ à instalação em que ele está associado seja minimizado. Em outras palavras, queremos encontrar $F' \subseteq F$ e uma função $\sigma$ que minimize $\sum_{i\in F'} f_i + \sum_{j \in D} c_{\sigma(j)j}$. 

O problema $k$-mediana é muito parecido com o problema de localização de instalações. A diferença aqui é que não temos custo para a abertura de instalações e podemos abrir no máximo $k$ delas.
 
Assim como no localização de instalações, no problema $k$-mediana temos um grafo $(F,D)$-bipartido completo em que $D$ é o conjunto de clientes a serem atendidos e $F$ o conjunto de instalações que podem ser abertas. Para cada cliente $j \in D$ e cada instalação $i \in F$, há um custo $c_{ij}$ para a aresta $ij$ em associar o cliente $j$ à instalação $i$. Temos também um inteiro $k$ que representa a quantidade de instalações que podem ser abertas. Então, queremos encontrar um conjunto $F' \subseteq F$ de tamanho $k$ e uma função que associe cada cliente a uma instalação aberta tal que o custo total de associação seja minimizado.

No problema dos $k$-centros não existe essa diferença entre instalações que podem ser abertas e clientes, teremos cidades e escolheremos $k$ delas para construir instalações.

Então, temos um grafo $G(V,E)$ completo em que $V$ são cidades e temos um custo $c_{e}$ para associar as cidades que são extremos de $e$. Temos também um inteiro $k$ que representa a quantidade de cidades em que uma instalação será aberta. Cada cidade será associada a uma cidade com uma instalação aberta com menor custo de associação entre elas. O objetivo do nosso problema é minimizar o maior custo de associação entre uma cidade qualquer e a cidade a qual ela está associada.



Diversos métodos podem ser utilizados para aproximar o problema de localização de instalações. Charikar e Guha desenvolveram um algoritmo com razão de aproximação $2.414$ utilizando o método de busca local~\cite{Charikar&Guha'05}.  Esse problema também pode ser modelado como um problema de programação inteira e, por isso, técnicas envolvendo programação linear podem ser aplicadas a ele.  Por exemplo, há algoritmos que fazem o arredondamento de soluções da relaxação linear do programa inteiro para obter uma solução aproximada do problema.  Alguns destes algoritmos atingem boas razões de aproximação, por exemplo, chegando a 1.677~\cite{Byrka&Aardal'10}. Entretanto, a melhor aproximação encontrada utiliza de vários métodos, incluindo o conhecido método primal-dual, e garante uma razão de aproximação 1.488~\cite{LI'13}. Essa não é muito distante do melhor que se poderia encontrar, uma vez que Guha e Khuller mostraram que não existe algoritmo para esse problema com razão de aproximação melhor que 1.463~\cite{GUHA1999228}, a menos que $P = \NP$.


Dentre os três problemas apresentados, o $k$-mediana é o que tem a maior folga entre o melhor resultado de inaproximabilidade e a razão do melhor algoritmo de aproximação conhecido. Jain, Mahdian e Saberi provaram que não existe algoritmo polinomial com razão de aproximação $1+ \frac{2}{e}$ para o $k$-mediana~\cite{JMS'02}, assumindo que $P \neq \NP$, enquanto a melhor aproximação encontrada tem razão $2.675 + \epsilon$~\cite{BPRST'17}.


Hsu e Nemhauser~\cite{HSU1979209} mostraram que não existe algoritmo polinomial com razão de aproximação menor que 2 para o problema dos $k$-centros, assumindo que $P\neq\NP$. Neste caso, temos algoritmos que apresentam o melhor desempenho possível: utilizando o método do gargalo, Gonzalez~\cite{GONZALEZ1985293} e independentemente Hochbaum e Shmoys~\cite{HochShmoys'85} desenvolveram um algoritmo polinomial com razão de aproximação igual a 2.
% O primeiro deles, o $k$-centros, é um problema clássico de otimização combinatória. Dadas $n$ cidades com distâncias especificadas entre elas e um número $k$, o problema quer determinar qual o melhor jeito de construir $k$ depósitos dentre um conjunto determinado das cidades tais que a maior distância entre uma cidade e o depósito mais próximo seja a menor possível.

% Nosso problema é modelado como um grafo completo $G(V,E)$, onde $V$ são as cidades e temos uma função distância~$d$ em que $d(e)$ representa a distância das cidades ligadas pela aresta $e$. O objetivo é encontrar um subconjunto $C \subseteq V$ de tamanho $k$ que minimize $\max_{v\in V}d(v,C)$ sendo $d(v,C) = \min_{u\in C}d(vu)$.

% Trabalharemos aqui com a versão métrica do problema em que a função $d$ obedece a desigualdade triangular, ou seja, $d(uv) \leq d(ux) + d(xv)$ para todas as triplas de vértices $u$, $x$, $v$. Veremos que, para qualquer função polinomialmente computável~$\alpha(n)$, não existe $\alpha(n)$-aproximação para a versão não métrica desse problema, assumindo $P\neq\NP$. Hsu e Nemhauser~\cite{HSU1979209} mostraram que não existe algoritmo polinomial com razão de aproximação menor que 2 para o problema do $k$-centros, assumindo $P\neq\NP$. Assim, temos algoritmos que apresentam o melhor desempenho possível: utilizando o método do gargalo, Gonzalez~\cite{GONZALEZ1985293} e independentemente Hochbaum e Shmoys~\cite{HochShmoys'85} desenvolveram um algoritmo polinomial com razão de aproximação igual a 2. 


% Localização de instalações é um problema clássico de otimização que determina a melhor localização para instalações, como fábricas ou depósitos, com base em demandas geográficas, custos de abertura de instalações e distâncias de transporte. Além disso, eles podem ser modelados para outras aplicações como problemas de posicionamento de caches em um computador ou problemas de projeto de redes.

% Existem várias versões do problema de localização de instalações. O mais simples deles é o de localização de instalações sem capacidades, ou seja, em que as instalações não tem limitações para suprir clientes.



% Diversos métodos podem ser utilizados para aproximar o problema de localização de instalações. Charikar e Guha desenvolveram um algoritmo com razão de aproximação $2.414$ utilizando o método de busca local~\cite{Charikar&Guha'05}.  Esse problema também pode ser modelado como um problema de programação inteira e, por isso, técnicas envolvendo programação linear podem ser aplicadas a ele.  Por exemplo, há algoritmos que fazem o arredondamento de soluções da relaxação linear do programa inteiro para obter uma solução do problema.  Alguns destes algoritmos atingem boas razões de aproximação, por exemplo, chegando a 1.677~\cite{Byrka&Aardal'10}. Entretanto, a melhor aproximação encontrada utiliza de vários métodos, incluindo o primal-dual, e garante uma razão de aproximação 1.488~\cite{LI'13}. Essa não é muito distante do melhor que se poderia encontrar, uma vez que Guha e Khuller mostraram que não existe algoritmo para esse problema com razão de aproximação melhor que 1.463~\cite{GUHA1999228}, a menos que $P = \NP$.

% O problema $k$-mediana é muito parecido com o problema de localização de instalações. A diferença aqui é que não temos custo para a abertura de instalações e podemos abrir no máximo $k$ delas. Assim, como no $k$-centros também vamos modelar nosso problema como um grafo e vamos também trabalhar com a versão métrica do problema.

% Uma instância do $k$-mediana consiste em um grafo ($F,C$)-bipartido completo, sendo $F$ o conjunto das instalações e $C$ o conjunto das cidades, um número inteiro $k$ positivo que representa a quantidade de instalações que podem ser abertas e um custo $c_{ij}$ de conexão de cada cidade $j$ a cada instalação $i$. Devemos determinar um conjunto $I \subseteq F$ tal que $|I| \leq k$ de instalações a serem abertas e uma função que atribua cada cidade a uma instalação aberta tal que minimize o custo total de conexão.

% Dentre os três problemas apresentados, esse é o que tem a maior folga entre o melhor resultado de inaproximabilidade e a razão do melhor algoritmo de aproximação. Jain, Mahdian e Saberi provaram que não existe algoritmo polinomial com razão de aproximação $1+ \frac{2}{e}$ para o $k$-mediana ~\cite{JMS'02}, enquanto a melhor aproximação encontrada tem razão $2.675 + \epsilon$~\cite{BPRST'17}.

\newpage
\section{$k$-Centros}

Seja $I(G,c,k)$ uma instância do problema dos $k$-centros e $C \subseteq V$ uma solução viável para $I$. Vamos definir alguns termos que facilitarão as explicações seguintes. Os vértices de $C$ serão chamados \emph{centros de cluster}. Os vértices de $V$ serão particionados em $k$ conjuntos chamados \emph{clusters} e cada um deles terá exatamente um centro de cluster. Um vértice estará no mesmo cluster que um centro de cluster associado a ele. Cada cluster terá um \emph{raio} que é o maior custo entre o seu centro e um vértice qualquer dele. O nosso problema se resume a encontrar um conjunto $C$ que minimize o maior desses raios. Denotamos por raio$(C)$ o maior raio de um cluster induzido por $C$.\\
Antes de falarmos sobre algoritmos de aproximação para o problema dos $k$-centros, vamos mostrar que, assumindo $P\neq\NP$, não existe algoritmo polinomial que resolva nosso problema, ou seja, vamos mostrar que nosso problema é $\NP$-difícil. Para isso, vamos definir o problema do $k$-conjunto dominante.

\begin{definition}
    Seja $G = (V,E)$ um grafo. Um conjunto $D \subseteq V$ é chamado \emph{dominante} se, para todo vértice $u \in V \setminus D$, existe um vértice $v \in D$ tal que $uv \in E$.
\end{definition}

\begin{problem}[$k$-conjunto dominante]
    Dado um grafo $G$ e um inteiro $k$, decidir se $G$ tem um conjunto dominante $D$ tal que $|D| \leq k$.      
\end{problem}
Esse problema é $\NP$-completo, sendo o problema GT2 do famoso livro de Garey e Johnson~\cite{garey1979computers}. Usaremos este problema para mostrar que o problema dos $k$-centros é $\NP$-difícil.

\begin{theorem}\label{theorem:2.3}
    O problema dos $k$-centros para instâncias métricas é $\NP$-difícil.
\end{theorem}

\begin{proof}
    Suponha que exista um algoritmo $A$ que resolve o problema dos $k$-centros em tempo polinomial. Seja $G = (V,E)$ um grafo e $I(G,k)$ uma instância do problema $k$-conjunto dominante. Vamos criar uma instância $I'(G',c,k)$ do problema dos $k$-centros a partir da instância $I$. A instância $I'$ tem como grafo $G'(V,E')$ completo tal que, para todo $e \in E'$, \\
    \[
    c_e = \begin{cases}
            1, \text{ se } e \in E \\
            2, \text{ caso contrário.} 
            \end{cases}\]\\
    Note que $c$ satisfaz a desigualdade triangular e pode ser obtida de $I$ em tempo polinomial.\\
    O algoritmo aplicado à instância $I'$ encontra uma solução $C$, ou seja, um conjunto de $k$ centros de cluster. Se raio$(C)=1$ então todos os vértices estão ligados ao centro do seu cluster com uma aresta de $G$ e assim $C$ é um conjunto dominante em $G$. \\
    Como o algoritmo $A$ minimiza o raio de $C$, se raio$(C)=2$, não existe uma solução para $I'$ em que os vértices estejam ligados ao centro do seus clusters apenas por arestas de $G$ e, por isso, não existe um conjunto dominante de tamanho menor ou igual a $k$ em $G$.

    Portanto, conseguimos resolver em tempo polinomial o problema do $k$-conjunto dominante, o que implicaria que $P = \NP$.
\end{proof}

O resultado acima pode ser adaptado para dar um resultado mais forte de inaproximabilidade para a versão geral do problema, não restrita à métrica.
\begin{theorem}
    Seja $\alpha(n)$ uma função computável com $\alpha(n)\geq 1$ para todo $n$. Não existe $\alpha(n)$-aproximação para a versão geral do $k$-centros, onde $n$ é o número de vértices do grafo da instância, a menos que $P=\NP$.
\end{theorem}

\begin{proof}
        A demonstração desse teorema é muito parecida com a do Teorema~\ref{theorem:2.3}. \\
        Seja $G = (V,E)$. Suponha que exista um algoritmo polinomial $A$ que é uma $\alpha(n)$-aproximação do $k$-centros e seja $I(G,k)$ uma instância do problema do $k$-conjunto dominante em que $G$ tem $n$ vértices. Vamos criar uma instância $I'(G',c,k)$ do problema dos $k$-centros a partir da instância $I$. A instância $I'$ tem como grafo $G'(V,E')$ completo tal que, para cada $e \in E'$, \\
    \[c_e = \begin{cases}
            1, \text{ se } e \in E \\
            \alpha(n)+1, \text{ caso contrário.} 
            \end{cases}\]\\
    Como $\alpha(n)$ é uma função computável, essa instância pode ser construída a partir de $I$ em tempo polinomial. \\
    Se $\alpha(n)=1$, então $c$ obedece a desigualdade triangular e $A$ é um algoritmo polinomial e exato, o que, pelo Teorema~\ref{theorem:2.3}, é absurdo. Então, suponha $\alpha(n)>1$. \\
    O algoritmo aplicado à instância $I'$ encontra uma solução $C$ de tamanho $k$. Como $c_e = 1$ ou $\alpha(n)+1$ para todo $e \in E'$, então raio$(C)=1$ ou $\alpha(n)+1$.\\
    Se raio$(C)=1$ então todos os vértices estão ligados ao centro do seu cluster com aresta de $G$, e assim $C$ é um conjunto dominante em $G$. \\
    Se raio$(C) = \alpha(n) + 1$, então $\opt(I) \geq \frac{\alpha(n)+1}{\alpha(n)}>1$. Assim, não existe solução $C'$ tal que raio$(C')=1$ e não existe $k$-conjunto dominante em $G'$ que utilize apenas arestas de $G$.
    Portanto, conseguimos resolver o problema do $k$-conjunto dominante em tempo polinomial o que, assumindo que $P \neq \NP$, é um absurdo.
\end{proof}
    Fica, então, explícita a impossibilidade de encontrarmos algoritmos de aproximação para a versão geral do problema dos $k$-centros.\\
    Agora que justificamos o estudo de algoritmos de aproximação para esse problema vamos mostrar um limitante inferior para a razão de aproximação desses algoritmos.
    
    \begin{theorem}
        Seja $\varepsilon \in (0,1]$. Não existe $(2-\varepsilon)$-aproximação para o problema dos $k$-centros, a menos que $P=\NP$.
    \end{theorem}
    A prova do Teorema~\ref{theorem:2.3} essencialmente serve também para esse Teorema, a única diferença é que aqui precisamos assumir a existência de um algoritmo que é uma $(2 - \varepsilon)$-aproximação para o problema dos $k$-centros.

    Começaremos por um algoritmo simples, mas que garante a melhor razão de aproximação.
    
\newpage
\subsection{Algoritmo guloso}
    Seja $G = (V,E)$ o grafo da instância do problema e $c_e$ o custo de uma aresta $e \in E$.
    Definimos a função $c(u,S)$ que é definida para um vértice $u$ e um conjunto $S$ de vértices  por,
    \[ c(u,S) = \min_{v\in S} c_{uv}
        \]
    A ideia desse algoritmo guloso se concentrará em, a cada iteração, escolher o vértice $u$ tal que $c(u,C)$ seja o maior possível, sendo $C$ o conjunto dos centros de cluster escolhidos até aquele momento.

    \begin{algorithm}
		\begin{algorithmic}[1]
			\Function{G}{$G,c,k$}
			\State Escolha arbitrariamente $u \in V$.
            \State $C \gets \{u\}$.
            \While{$|C| \leq k$}
            \State $v \gets \arg\max_{j \in V} c(j,C)$
            \State $C \gets C \cup \{v\}$
            \EndWhile 
			\State Devolva $C$.
			\EndFunction
		\end{algorithmic}
	\end{algorithm}
    
    \begin{theorem}
        O algoritmo {\sc G}$(G,c,k)$ é uma $2$-aproximação do problema dos $k$-centros.
    \end{theorem}
    \begin{proof}
        Primeiramente, vamos mostrar que o algoritmo é polinomial. A única linha que não é claramente polinomial é a linha $5$. Para cada vértice $v \in V$, vamos olhar o custo de $|C|$ arestas. Como $|C| \leq |V|$, então iremos olhar o custo de no máximo $|V|^2$ arestas a cada iteração. Assim, nosso algoritmo roda em tempo $O(k|V|^2)$. \\
        Seja $C^*$ um conjunto ótimo de centros de cluster para $I(G,c,k)$. Vamos mostrar que raio$(C) \leq 2\opt(I) $. Perceba que o custo entre dois vértices quaisquer de um mesmo cluster de $C^*$ é no máximo $2\opt(I)$, uma vez que vale a desigualdade triangular.\\ 
        Se há um vértice de $C$ em cada cluster de $C^*$, todos os vértices estão ligados com uma aresta de custo no máximo $2\opt(I)$ a um centro de cluster de $C$. Então raio$(C) \leq 2\opt(I)$. \\
        Senão, suponha sem perda de generalidade que $C_{i-1} = \{ u_1,u_2,\ldots,u_{i-1}\}$ é o $C$ ao final da iteração $i-1$ e cada $u_j$ está em um cluster diferente de $C^*$. Seja $u_i$ o vértice escolhido na iteração $i$ e suponha que ele está no mesmo cluster que um vértice $u_j$ para algum $j=1,\ldots,i-1$. Então, $c(u_i,C_{i-1}) \leq c(u_iu_j) \leq 2\opt(I)$. Como $u_i$ maximiza $c(v,C_{i-1})$, então raio$(C) \leq 2\opt(I)$.
    \end{proof}
    Esse algoritmo é de autoria de Gonzalez~\cite{GONZALEZ1985293}.
    Vamos mostrar que essa análise é justa, ou seja, existe uma instância $I(G,c,k)$ em que o algoritmo devolve uma solução $C$ tal que raio$(C) = 2 \opt(I)$. \\
    O grafo $G$ contém pelo menos $k+2$ vértices e as arestas têm custo 1 ou 2. O grafo induzido pelas arestas de custo 1 é uma estrela, como mostrado na figura abaixo.
    \[
    \begin{tikzpicture}
        % Number of outer vertices in the star
        \def\n{8}
        
        % Radius of the star
        \def\r{2cm}
        
        % Rotation angle
        \def\rotationangle{90}
        
        % Draw the central vertex
        \draw (0,0) node[circle, draw, minimum size=10pt, inner sep=1.5pt] (center) {};
        
        % Draw the outer vertices, rotated
        \begin{scope}[rotate=\rotationangle]
          \foreach \i in {1,...,\n} {
            \draw (\i*360/\n:\r) node[circle, draw, minimum size=10pt, inner sep=1.5pt] (\i) {};
          }
        \end{scope}
        
        % Connect the central vertex to every outer vertex
        \foreach \i in {1,...,\n} {
          \draw (center) -- (\i);
        }
        \draw (4) node[shift={(0.75,0.2)}] {\rotatebox{30}{\scalebox{1.5}{$\ldots$}}} (4);
      \end{tikzpicture}
      \] \\
      Claramente, o raio de uma resposta ótima dessa instância é 1 e inclui o vértice do centro da estrela como um dos centros de cluster. \\
      Note que se, no algoritmo guloso, o vértice escolhido arbitrariamente for algum dos vértices da ponta dessa estrela o vértice do centro nunca será escolhido, uma vez que ele nunca maximizará a função $c(j,C)$. Assim, serão escolhidos apenas vértices da ponta da estrela e como temos pelo menos $k+1$ delas, sempre existirá um vértice ligado ao centro do seu cluster com uma aresta de custo 2.
\newpage



\subsection{Método do gargalo}
Os chamados problemas de gargalo são aqueles definidos em grafos com pesos nas arestas tais que a resposta ótima é o peso de uma aresta. \\
Para esse algoritmo será necessário saber o que é um conjunto independente de vértices.
\begin{definition}
    Seja $G = (V,E)$ um grafo. Um conjunto $S \subseteq V$ é um conjunto \emph{independente} se não existe $e \in E$ que tenha ambos os extremos em $S$.
\end{definition}
Seja $I(G,c,k)$ uma instância do problema dos $k$-centros. Vamos supor que as arestas $E = \{e_1,e_2,\ldots,e_{|E|}\}$ de $G$ estejam dispostas de forma que $c_{e_i} \leq c_{e_{i+1}}$ para todo $i \in [|E|-1]$. Como sabemos que é possível ordenar em tempo polinomial, podemos assumir isso. \\
Seja $E_i = \{e_1,e_2,\ldots,e_i\}$ e $G_i = (V,E_i)$. Seja também $i^*$ o menor $i$ tal que $G_i$ tem um $k$-conjunto dominante. Como $G$ é completo, $i^*$ existe. Claramente $c_{e_{i^*}} = \opt(I)$, porém não conseguimos encontrar $i^*$ eficientemente, uma vez que não é possível saber se um grafo tem um $k$-conjunto dominante em tempo polinomial, a menos que $P = \NP$. Portanto, vamos usar um conjunto independente maximal para aproximar uma resposta.

\begin{lemma}\label{lemma:2.8}
    Seja $G = (V,E)$ um grafo. Um conjunto independente maximal em $G$ é também um conjunto dominante.
\end{lemma}
\begin{proof}
    Seja $G = (V,E)$ e $S$ um conjunto independente maximal em $G$. Suponha, por absurdo, que $S$ não é um conjunto dominante. Então, existe vértice $u \in V \setminus S$ que não é vizinho de nenhum dos vértices de $S$. Portanto, $S \cup \{u\}$ é também um conjunto independente e $S \subset \{S \cup \{u\}\}$. Contradição, pois $S$ é maximal.
\end{proof}
Então, se encontrarmos um conjunto independente maximal de tamanho $k$ em $G$ teremos um conjunto dominante de mesmo tamanho. No entanto, não conseguimos garantir que iremos encontrar esse conjunto em $G$ e, por isso, vamos definir e usar o seu quadrado.
\begin{definition}
    Seja $G= (V,E)$ um grafo. Denotamos por $G^2 = (V,E^2)$ o \emph{quadrado} de $G$ em que $E^2 = E \cup \{uv: \text{distância de u a v em $G$ é $2$}\}$.
\end{definition}
Dada a definição vamos enunciar e provar um lema que nos ajudará no algoritmo.
\begin{lemma}\label{lemma:2.10}
    Seja $G$ um grafo, $G^2$ o seu quadrado e $k$ um inteiro positivo. Se $G$ contém um $k$-conjunto dominante então todo conjunto independente maximal em $G^2$ tem tamanho no máximo~$k$.
\end{lemma}
\begin{proof}
    Vamos mostrar pela contrapositiva.\\
    Seja $G = (V,E)$ um grafo, $G^2$ o seu quadrado e $k$ um inteiro positivo. Suponha que $G^2$ tem um conjunto independente maximal $S \subset V$ tal que $|S| > k$. Seja $D$ um conjunto dominante em $G$, vamos mostrar que $|D| > k$. \\
    Por definição, não existe um caminho de tamanho $2$ ou uma aresta entre dois vértices de $S$ em $G$. Seja $u \in D$, vamos mostrar que $u$ cobre no máximo $1$ vértice de $S$ em $G$. \\
    Se $u \in S$, o único vértice de $S$ coberto por $u$ é ele mesmo, uma vez que não existe aresta entre dois vértices de $S$ em $G$. \\
    Se $u \not \in S$, $u$ pode cobrir apenas um vértice de $S$, caso cobrisse $2$, digamos $v$ e $w$, então $P = (vu,uw)$ é um caminho de tamanho $2$ com extremos em $v$ e $w$ em $G$. Assim, $vw$ é uma aresta em $G^2$ e eles não podem pertencer a $S$ ao mesmo tempo. \\
    Portanto, $|D| \geq |S| > k$.
\end{proof}
Agora, temos todas as definições e lemas que serão necessários para nosso algoritmo.
\begin{algorithm}
    \begin{algorithmic}[1]
        \Function{GHS}{$G,c,k$}
        \State $i \leftarrow 0$
        \State $M_0 \leftarrow V$
        \While{$|M_i| > k$}
            \State $i\leftarrow i + 1$
            \State Seja $M_i$ um conjunto independente maximal em $G_i^2$
        \EndWhile
        \State Devolva $M_i$
        \EndFunction
    \end{algorithmic}
\end{algorithm}

\begin{theorem}
    O algoritmo {\sc GHS}$(G,c,k)$ é uma $2$-aproximação do problema dos $k$-centros.
\end{theorem}
\begin{proof}
    Primeiro vamos mostrar que o algoritmo é polinomial. \\
    Como $G_{i^*}$ tem um $k$-conjunto dominante, então o laço vai iterar no máximo $i^*$ vezes, pois pelo Lema~\ref{lemma:2.10} qualquer conjunto independente maximal encontrado em $G_{i^*}^2$ terá tamanho no máximo $k$.
    Também é fácil mostrar que é possível encontrar um conjunto independente maximal em tempo polinomial. Um algoritmo simples que começa com um conjunto $A = \{u\}$ sendo $u$ um vértice arbitrário e, a cada iteração, é colocado em $A$ um vértice que não é adjacente a nenhum vértice de $A$ até não ser mais possível é suficiente. \red{Faço um lema para explicar melhor?}\\
    Além disso, também conseguimos construir o grafo $G_i^2$ em tempo polinomial. Começaremos $E_i^2$ como uma cópia de $E_i$ e, para cada tripla de vértice $(u,v,k)$ caso já não exista uma aresta $uk \in E_i$, vamos inseri-la em $E_i^2$ se $uv,vk \in E_i$. Como temos no máximo $|V|^3$ triplas de vértices e todas as operações que serão feitas são feitas em tempo polinomial, então podemos construir $G_i^2$ em tempo polinomial. \\
    Agora, vamos mostrar que é uma $2$-aproximação. \\
    Seja max$(H)$ o maior valor de uma aresta em um grafo $H$ com peso nas arestas. Seja $i'$ o valor de $i$ ao final do algoritmo e $M_{i'}$ a solução devolvida por ele. Como $M_{i'}$ é um conjunto independente maximal de tamanho no máximo $k$, então pelo Lema~\ref{lemma:2.8} ele é um $k$-conjunto dominante e como $M_{i'}$ é um subgrafo de $G_{i'}^2$ então max$(M_{i'}) \leq \text{max}(G_{i'}^2) $. Pela desigualdade triangular, é fácil notar que max$(G_{i'}^2) \leq 2$  max$(G_{i})$. \\Assim, max$(M_{i'}) \leq $ max$(G_{i'}^2) \leq 2$ max$(G_{i'}) \leq 2$ max$(G_{i^*})= 2 \opt(I)$. 
\end{proof}

Este algoritmo é de autoria de Gonzalez~\cite{GONZALEZ1985293} e independentemente Hochbaum e Shmoys~\cite{HSBottle}, e, por isso, está sendo creditado com suas iniciais em seu nome. 
\newpage
\section{Localização de instalações}
Antes de falarmos sobre algoritmos de aproximação para o problema da localização de instalações, vamos mostrar que, assumindo $P\neq\NP$, não existe algoritmo polinomial que resolva nosso problema, ou seja, vamos mostrar que nosso problema é $\NP$-difícil. Para isso, vamos definir o problema da cobertura por vértices.



\begin{problem}(Cobertura por vértices)
    Dado um grafo $G$ e um inteiro $k$, decidir se $G$ tem uma cobertura por vértices de tamanho $k$.
\end{problem}
Esse problema é $\NP$-completo, sendo um dos famosos $21$ problemas do Karp~\cite{Karp1972}. Usaremos este problema para mostrar que o problema localização de instalações é $\NP$-difícil.

\begin{theorem}
    O problema localização de instalações é $\NP$-difícil.
\end{theorem}

\begin{proof}
    Suponha que exista um algoritmo $A$ que resolva o problema localização de instalações em tempo polinomial. \\
    Seja $I(G,k)$ uma instância do problema da cobertura por vértices em que $G = (V,E)$. Vamos criar uma instância $I'(F,D,c,f)$ do problema de localização de instalações a partir de $I$.\\
    O conjunto de possíveis instalações a serem abertas $F$ será o conjunto de vértices $V$. O conjunto de clientes $D$ será o conjunto de arestas $E$. O custo de abertura $f_i$ será 1 para cada instalação $i$. O custo de associação $c_{ij} \text{, para } i \in F \text{ e } j \in D$, será 0 se o vértice referente a $i$ é extremo da aresta referente a $j$ e 2 caso contrário. \\
    A resposta ótima será a menor quantidade de instalações que precisamos abrir até que todos os clientes possam ser associados a instalações em que o seu custo de associação é 0. Se um cliente está associado a uma instalação cujo custo de associação é 2, podemos diminuir o custo total abrindo uma instalação em que o custo de sua associação é 0, uma vez que abrir a instalação custaria 1 e diminuiríamos o custo de associação do cliente de 2 para 0. Sabemos que essa instalação existe, pois para cada cliente existem duas instalações com custo de associação 0. \\
    Assim, é fácil notar que esse é o tamanho mínimo de uma cobertura por vértices em $G$. Portanto, se o custo total da solução do algoritmo $A$ aplicado a instância $I'$ é menor ou igual a $k$, então a resposta para $I$ é sim. Caso contrário, a resposta é não. \\
    Desse modo, resolvemos o problema da cobertura por vértices em tempo polinomial, o que, assumindo $P\neq\NP$, é um absurdo.
\end{proof}
\subsection{Algoritmos baseados em programação linear}
Nessa seção vamos mostrar dois algoritmos para o problema de localização de instalações: o arredondamento em programa linear e um algoritmo que utiliza o método primal-dual. \\
Para isso, vamos modelar o nosso problema como um programa linear inteiro. Vamos relaxá-lo e mostrar o seu dual, que será usado em ambos os algoritmos. \\
Nosso programa inteiro terá dois tipos de variáveis: uma variável $y_i$ para cada $i \in F$ que terá valor 1 se a instalação $i$ foi aberta e 0 caso contrário, e uma variável $x_{ij}$ para cada $i \in F$ e $j \in D$ que terá valor 1 se o cliente $j$ estiver associado a instalação $i$ e 0 caso contrário. Então o programa inteiro para a instância $I(F,D,c,f)$ é:
\begin{align}
 \text{Minimizar} \quad & \sum_{i \in F}f_iy_i + \sum_{i\in F,j\in D}c_{ij}x_{ij} \\
 \text{sujeito a} \quad & \sum_{i\in F} x_{ij}=1, \quad \forall j \in D \\
 &x_{ij} \leq y_i,\quad \quad \; \; \forall i\in F,j\in D\\
 &x_{ij} \in \{0,1\} ,\quad \forall i\in F,j\in D\\
 &y_i \in \{0,1\}, \quad \; \,\forall i\in F.
\end{align}
\\
Para a relaxação, apenas impomos que $x_{ij} \geq 0$ e $y_i \geq 0$ no lugar de (4) e (5), respectivamente. Assim, temos o seguinte primal e dual correspondente à nossa relaxação:
    \begin{align}
        \text{Minimizar} \quad & \sum_{i \in F}f_iy_i + \sum_{i\in F,j\in D}c_{ij}x_{ij} \tag{P1} \label{P1}\\
        \text{sujeito a} \quad & \sum_{i\in F} x_{ij}=1,  \forall j \in D \tag{P2} \label{P2}\\
        &x_{ij} \leq y_i, \, \; \; \forall i\in F,j\in D \tag{P3} \label{P3}\\
        &x_{ij} \geq 0,\quad \forall i\in F,j\in D\tag{P4}\label{P4}\\
        &y_i \geq 0, \quad \; \,\forall i\in F \tag{P5} \label{P5}
       \end{align}

    \begin{align}
        \text{Maximizar} \quad & \sum_{j \in D} v_{j} \tag{D1} \label{D1}\\
        \text{sujeito a} \quad & \sum_{j\in D} w_{ij}\leq f_i, \quad \forall i \in F \tag{D2} \label{D2}\\
        &v_{j} - w_{ij}\leq c_{ij},  \; \; \forall i\in F,j\in D \tag{D3} \label{D3}\\
        &w_{ij} \geq 0 ,\quad \forall i\in F,j\in D\tag{D4} \label{D4}\\
        &v_j \geq 0, \quad \; \,\forall j\in D \tag{D5} \label{D5}.
       \end{align}
Sabemos que cada desigualdade do primal tem uma desigualdade correspondente no dual, assim temos que os seguintes pares de desigualdades correspondentes: (\ref{P2},\ref{D5}), (\ref{P3},\ref{D4}), (\ref{P4},\ref{D3}) e (\ref{P5},\ref{D2}).

Como o primal é uma relaxação do nosso problema vale que o valor de uma solução ótima dele é no máximo o ótimo do nosso problema.


\newpage
\subsubsection{Arredondamento do programa linear}
Vamos assumir aqui que a função custo satisfaz a desigualdade triangular. Como a função custo só está definida de um cliente para uma instalação, então a desigualdade triangular terá uma cara diferente. Para clientes $j,l$ e  instalações $i,k$, então
\[c_{ij} \leq c_{il} + c_{kl} + c_{kj}.\]
Vamos então definir algumas coisas que serão necessárias no nosso algoritmo.
\begin{definition}
    Seja $(x^*,y^*)$ uma solução do programa linear. Dizemos que uma instalação $i$ está na \emph{vizinhança} de um cliente $j$ se $x^*_{ij} > 0$. Seja $N(j) = \{ i \in F : x^*_{ij} > 0\}$.
\end{definition}
Dada essa definição, vamos enunciar um lema que irá nos ajudar a provar uma cota superior para o custo de associação da solução aproximada escolhida pelo algoritmo.
\begin{lemma}\label{lemma:3.5}
    Sejam $(x^*,y^*)$ e $(v^*,w^*)$ soluções do programa linear e do seu dual, respectivamente, e $j$ um cliente qualquer. Para todo $i \in N(j)$, $c_{ij} \leq v^*_j$.
\end{lemma}
\begin{proof}
    Como $i \in N(j)$, então $x^*_{ij}>0$. Assim, pelas folgas complementares, a desigualdade do dual correspondente a~\ref{P4} vale por igualdade, então $v^*_j - w^*_{ij} = c_{ij}$. Como $w^*_{ij} \geq 0$, $c_{ij} \leq v^*_j$. 
\end{proof}
Com esse lema, sabemos que se em um conjunto $S \subseteq F$ de instalações abertas, para todo cliente $j \in D$, existe uma instalação aberta $i$ em $N(j)$, então $c_{ij}\leq v_j^*$ e o custo total de associação seria no máximo $\sum_{j\in D}v_j^* \leq \opt(I)$, uma vez que vale a dualidade forte e que o valor do primal é no máximo $\opt(I)$. Entretanto, não é garantido uma cota superior para o custo de abertura de $S$. Vamos descobrir como encontrar um $S$ com um bom custo de abertura e um bom custo de associação. Seja $j_k$ um cliente qualquer. Se abrirmos a instalação $i_k$ mais barata de $N(j_k)$ conseguimos limitar o custo de sua abertura.
\[\tag{*} \label{relx_fl:*}
    f_{i_k} = f_{i_k} \sum_{i \in N(j_k)}x^*_{ij_k} = \sum_{i \in N(j_k)}f_{i_k}x^*_{ij_k} \leq \sum_{i \in N(j_k)}f_{i}y^*_{i},
\]
onde a primeira igualdade vale por \ref{P3} e pela definição de $N(j_k)$ e a desigualdade vale por termos escolhido $i_k$ como a instalação mais barata de $N(j_k)$.

Essa informação será importante para a prova da razão de aproximação do nosso algoritmo. Agora, faremos uma última definição necessária para o nosso algoritmo.

\begin{definition}
    Seja $j\in D$ um cliente. Seja $N^2(j)$ o conjunto dos clientes $i \in D$ tais que $N(i) \cap N(j) \neq \emptyset$.
\end{definition}
Agora, vamos definir o algoritmo. Nele, $S$ será o conjunto das instalações a serem abertas e $\sigma : D \rightarrow F $ a função que associa cada cliente a uma instalação aberta. Ambos serão montados durante o algoritmo.
\begin{algorithm}
    \begin{algorithmic}[1]
        \Function{CS}{$F,D,c,f$}
        \State Sejam $(x^*,y^*)$ e $(v^*,w^*)$ soluções ótimas para o programa linear e o seu dual
        \State $S \gets \emptyset$
        \State $C \gets D$ 
        \State $k \gets 0$
        \While{$C \neq \emptyset$}
        \State Escolha $j_k \in C$ que minimize $v_j^*$
        \State Escolha $i_k \in N(j_k)$ que minimize $f_{i}$
        \State $S \gets S \cup \{i_k\}$
        \For{\text{todo $j \in N^2(j_k) \cap C$}}
        \State $\sigma(j) \leftarrow i_k$
        \EndFor
        \State $C \gets C \setminus N^2(j_k)$
        \State $k \gets k+1$
        \EndWhile
        \State Devolva $S$ e $\sigma$
        \EndFunction
    \end{algorithmic}
\end{algorithm}

\begin{theorem}
    O algoritmo {\sc CS} é uma $4$-aproximação para o problema da localização de instalações.
\end{theorem}
\begin{proof}
    Primeiro, vamos mostrar que o algoritmo é polinomial. Sabemos que é possível encontrar uma solução para o problema linear e o seu dual em tempo polinomial utilizando o algoritmo dos elipsoides~\cite{Kha79}. Sabemos que o laço da linha 6 irá executar no máximo $|D|$ iterações, uma vez que sempre retiramos pelo menos um elemento de $C$. Além disso, as linhas $6-11$ são claramente polinomiais.
    Agora, vamos mostrar que o algoritmo é uma $4$-aproximação.\\
    Perceba que, para $k_1$ e $k_2$ com $k_1 < k_2$, $N(j_{k_1})\cap N(j_{k_2}) = \emptyset$, caso contrário $j_{k_2} \in N^2(j_{k_1})$ e seria retirado de $C$ na iteração $k_1$.\\
    Seja $F' \subseteq F$, tal que $F' = \bigcup_k N(j_k)$.
    Por (\ref{relx_fl:*}), vale que $f_{i_k} \leq \sum_{i \in N(j_k)}f_{i_k}y^*_{i}$. Então se somarmos para todo $k$, temos
    \[ \sum_kf_{i_k} \leq \sum_k \sum_{i \in N(j_k)}f_{i}y^*_{i} = \sum_{i \in F'}f_{i}y^*_{i} \leq \sum_{i \in F}f_{i}y^*_{i} \leq \opt(I)\]
    Agora, vamos fixar uma iteração $k$ e sejam $i = i_k$ e $j = j_k$. Denotemos por $C_k$ o conjunto $C$ no início da iteração $k$. Pelo Lema~\ref{lemma:3.5}, sabemos que $c_{ij} \leq v^*_j$. Seja $\ell \in N^2(j) \cap C_k \setminus \{j\}$ e $h \in N(\ell) \cap N(j)$. Pela desigualdade triangular e como $j$ minimiza $v^*_j$ em $C_k$, temos
    \[ c_{i\ell} \leq c_{ij} + c_{hj} + c_{h\ell} \leq v_j^* + v_j^* + v_{\ell}^* \leq 3 v_{\ell}^*
        \]
    Isso vale para todo $\ell \in D$, então
    \[\sum_{j\in D} c_{\sigma(j)j} \leq 3 \sum_{j \in D} v^*_j \leq 3\opt(I),\]
    onde a segunda desigualdade vale pela dualidade forte e pela relaxação. \\
    Assim, temos que o custo de abertura das instalações é no máximo $\opt(I)$ e o custo de associação é no máximo $3\opt(I)$. Portanto, o custo total da solução $S$ é no máximo $4~\opt(I)$.
\end{proof}
O algoritmo aqui apresentado é de autoria de Chudak e Shmoys~\cite{Chudak2003}.

\newpage
\subsubsection{Método primal-dual}

\newpage

\subsection{Busca local}
Embora o custo de transporte esteja definido apenas para um cliente e uma instalação, aqui falaremos também sobre custos de transporte entre duas instalações. Podemos definir o custo de transporte entre duas instalações como o menor custo de transporte entre essas duas instalações e um cliente qualquer. \\
Um algoritmo de busca local começa com uma solução viável para o problema e checa se alguma alteração local melhora o custo da solução atual. Caso essa melhora ocorra, essa alteração é feita. Esse processo se repete até que não exista alteração que melhore o custo e, então, encontramos uma solução que é chamada de localmente ótima. A implementação direta da ideia da busca local na maioria das vezes não garante execução em tempo polinomial e, nesses casos, as alterações só acontecerão se satisfizerem alguma restrição. \\
Nosso algoritmo contará com três operações: abrir instalações fechadas, fechar instalações abertas ou trocar uma instalação aberta por uma fechada. Nossa solução inicial terá todas as instalações abertas e um cliente sempre estará ligado à instalação aberta mais próxima a ele. Uma operação só será feita se melhorar o custo da solução atual em uma razão de $(1-\delta)$, como essa solução não é necessariamente localmente ótima, iremos chamá-la de solução quase localmente ótima. Ao final, mostraremos que o algoritmo é uma $3(1 + \epsilon)$-aproximação para o problema de localização de instalações e vamos definir o $\delta$ em função do $\epsilon$. \\
Vamos mostrar duas desigualdades que serão fundamentais para chegar na razão de aproximação desejada. A partir daqui utilizaremos uma solução parcialmente localmente ótima $(S,\sigma)$ com custo das instalações abertas $A$ e custo de transporte total $T$. Também vamos utilizar uma solução ótima $(S^*,\sigma^*)$ com custos de instalações abertas $A^*$ e custo de transporte total $T^*$. Claramente, $\opt(I) = A^* + T^*$. Também definimos $m = |F|$ o número total de instalações.
\newpage
\begin{lemma}
    \label{lema:3.7}
    Sejam as soluções $(S,\sigma)$ e $(S^*,\sigma^*)$ como definidas acima. Então,
    \[ T - A^* - T^* \leq m \delta (A+T)\]
\end{lemma}
\begin{proof}
    Seja $i^* \in S^* \setminus S$. Como a solução $(S,\sigma)$ é quase localmente ótima, se abrimos $i^*$ e associar a ele em $\sigma$ os clientes que estão associados a ele em $\sigma^*$ não melhoraremos a solução em menos que $(1-\delta)$. Então,
    \begin{align*} A + f_{i^*} + T + \sum_{j : \sigma^*(j) = i^*} (c_{i^*j} - c_{\sigma(j)j}) &\geq (1-\delta)(A+T) \\
    f_{i^*} + \sum_{j : \sigma^*(j) = i^*} (c_{i^*j} - c_{\sigma(j)j}) &\geq -\delta(A+T)
    \end{align*} 
    Vamos agora, mostrar que essa desigualdade também é válida para $i^* \in S^* \cap S$. Como os clientes sempre estão ligados à instalação aberta mais próxima a eles, então o custo de trocar os clientes que estão associados a $i^*$ em $\sigma^*$ para $i^*$ em $\sigma$ é não negativo. Assim, 
    \begin{align*}
        f_{i^*} + \sum_{j : \sigma^*(j) = i^*} (c_{i^*j} - c_{\sigma(j)j}) &\geq \sum_{j : \sigma^*(j) = i^*} (c_{i^*j} - c_{\sigma(j)j}) \\
        &\geq 0 \: \: \geq -\delta(A+T) 
    \end{align*}

    Assim, podemos juntar essa desigualdade para todos as instalações em $S^*$,
    \begin{align*}
        \sum_{i^* \in S^*}\Big( f_{i^*} + \sum_{j : \sigma^*(j) = i^*} (c_{i^*j} - c_{\sigma(j)j})\Big) &\geq -m\delta(A+T) \\
        A^* + \sum_{j \in D} (c_{\sigma^*(j)j} - c_{\sigma(j)j}) & \geq -m\delta(A+T)\\
        A^* + T^* - T &\geq -m\delta(A+T)\\
        T - A^* - T^* &\leq m\delta(A+T)
    \end{align*}
\end{proof}
Para a segunda desigualdade, precisaremos de uma função e de um lema que ajudará a limitar o custo da redesignação de clientes. \\
Vamos definir a função $\phi : S^* \rightarrow S$ como \[\phi(i^*) = \arg\min_{i \in S} c_{i^*i},\] ou seja, a instalação $i$ em $S$ mais próxima a $i^* \in S^*$. Então, teremos o seguinte lema
\begin{lemma}
    \label{lemma:3.8}
    Seja $j \in D$ tal que $\sigma(j) = i$, $\sigma^*(j) = i^*$, $\phi(i^*) = i'$ e $i\neq i'.$ Então, \[c_{i'j} - c_{ij} \leq 2c_{i^*j}.\]
\end{lemma}

\begin{proof}
    Pela desigualdade triangular temos que:
    \[c_{i'j} \leq c_{i'i^*} + c_{i^*j}\]
    devido a definição de $i'$ temos que $c_{i'i^*} \leq c_{ii^*}$. Assim,
    \[c_{i'j} \leq c_{ii^*} + c_{i^*j}.\]
    Novamente, pela desigualdade triangular, temos que $c_{ii^*} \leq c_{ij} + c_{i^*j}$. Então,
    \begin{align*}
        &c_{i'j} \leq c_{ij} + 2 c_{i^*j}\\
        &c_{i'j} - c_{ij} \leq 2 c_{i^*j}
    \end{align*}
\end{proof}

Agora, conseguimos provar a segunda desigualdade que será necessária para a razão de aproximação.
\begin{lemma}
    \label{lema:3.9}
    Seja $(S,\sigma)$ e $(S^*,\sigma^*)$ como já definido. Então,
    \[A - A^* - 2T^* \leq m \delta(A+T)\]
\end{lemma}
\begin{proof}
    Antes de começar a demonstração em si, vamos definir o que é uma instalação segura. Uma instalação $i \in S$ é segura se e somente se não existe $i^* \in S^*$ tal que $\phi(i^*)=i$. \\
    Seja $i \in S$ uma instalação segura, como $(S,\sigma)$ é uma solução parcialmente localmente ótima, então se fecharmos $i$ e redesignarmos cada cliente j associado a $i$ para $\phi(\sigma^*(j))$ não melhoramos o custo da solução em uma razão melhor que $(1-\delta)$. Então,
    \begin{align*}
        A - f_i + T + \sum_{j:\sigma(j) = i} (c_{\phi(\sigma^*(j))j} - c_{\sigma(j)j}) &\geq (1-\delta)(A+T)\\
        - f_i + \sum_{j:\sigma(j) = i} (c_{\phi(\sigma^*(j))j} - c_{\sigma(j)j}) &\geq -\delta(A+T)
    \end{align*}
    Como $i$ é uma instalação segura o Lema~\ref{lemma:3.8} vale para todos os clientes que estão associados a $i$ em $\sigma$, então
    \begin{align} 
        \label{segura}
        - f_i + \sum_{j:\sigma(j) = i} 2c_{\sigma^*(j)j} &\geq -\delta(A+T)
    \end{align}
    Seja $i\in S$ uma instalação não segura. Vamos definir o conjunto $R \subseteq S^*$ como o conjunto das instalações que fazem com que $i$ não seja segura, ou seja, $R = \{i^* \in S^* : \phi(i^*) = i\}$. Vamos definir também $i' = \arg\min_{i^* \in R} c_{i^*i}$, a instalação de $R$ mais próxima de $i$. \\
    Seja $i^* \in R\setminus\{i'\}$, abrir $i^*$ e associar a $i^*$ os clientes que estão associados a $i$ em $\sigma $ e a $i^*$ em $\sigma^*$ não pode melhorar a resposta em uma razão melhor que $(1-\delta)$, assim
    \begin{align}
        A + f_i + T + \sum_{j: \sigma(j) = i \text{ \& } \sigma^*(j) = i^*}(c_{i^*j} - c_{ij}) &\geq (1-\delta)(A+T) \\
        \label{não segura}
        f_i + \sum_{j: \sigma(j) = i \text{ \& } \sigma^*(j) = i^*}(c_{i^*j} - c_{ij}) &\geq -\delta(A+T)
    \end{align}
    Agora vamos ver o que acontece se abrimos $i'$, fecharmos $i$ e associarmos os clientes $j$ associados a $i$ em $\sigma$ para $\phi(\sigma^*(j))$ se $\sigma^*(j) \not \in R$ e a $i'$ caso contrário.
    \begin{align*}
        - f_i + f_{i'} + \sum_{j: \sigma(j) = i \text{ \& } \sigma^*(j)\not \in R}(c_{\phi(\sigma^*(j)j)} - c_{ij}) + \sum_{j: \sigma(j)=i \text{ \& }\sigma^*(j) \in R}(c_{i'j} - c_{ij}) &\geq -\delta(A+T)
    \end{align*}
    No somatório dos clientes $j$ em que $\sigma^*(j) \not \in R$ podemos utilizar o Lema~\ref{lemma:3.8},
    \begin{align*}
        - f_i + f_{i'} + \sum_{j: \sigma(j) = i \text{ \& } \sigma^*(j)\not \in R} 2c_{\sigma^*(j)j} + \sum_{j: \sigma(j)=i \text{ \& }\sigma^*(j) \in R}(c_{i'j} - c_{ij}) &\geq -\delta(A+T)
    \end{align*}
    juntando essa desigualdade com \ref{não segura} para todas as instalações em $R\setminus\{i'\}$, temos
    \begin{align*}
        -f_i + \sum_{i^* \in R}f_{i^*} + \sum_{j: \sigma(j) = i \text{ \& } \sigma^*(j)\not \in R} 2c_{\sigma^*(j)j} + \sum_{j: \sigma(j)=i \text{ \& }\sigma^*(j) \in R}(c_{i'j} - c_{ij}) \\+ \sum_{j:\sigma(j)=i \text{ \& }\sigma^*(j) \in R \setminus\{i'\}}(c_{\sigma^*(j)j} - c_{ij}) \geq -|R|\delta(A+T)
    \end{align*}
    Vamos reduzir essa expressão. Seja $j$ tal que $\sigma(j)=i$ e $\sigma^*(j) \in R$. \\
    Se $\sigma^*(j) = i'$, $j$ só aparece no terceiro somatório e certamente $c_{i'j} - c_{ij} \leq 2 c_{i'j}$. \\
    Se $\sigma^*(j)\neq i'$, então na soma temos $c_{i'j} + c_{\sigma^*(j)j} - 2 \,c_{ij}$. Pela desigualdade triangular temos que $c_{i'j} \leq c_{i'i} + c_{ij},$ assim
    \[c_{i'j} + c_{\sigma^*(j)j} - 2 \,c_{ij} \leq c_{\sigma^*(j)j} + c_{i'i} - c_{ij} \]
    pela definição do $i'$ temos que $c_{i'i} \leq  c_{\sigma^*(j)i}$ e pela desigualdade triangular temos que $c_{\sigma*(j)i} \leq c_{\sigma^*(j)j} + c_{ij}$, então
    \[ c_{\sigma^*(j)j} + c_{i'i} -  c_{ij} \leq 2 c_{\sigma^*(j)j}\]
    Então, para qualquer $j$ temos que o custo referente a $j$ na soma é no máximo $2c_{\sigma^*(j)j}$. Assim,
    \[\sum_{i^* \in R}f_{i^*} - f_i + \sum_{j: \sigma(j)= i } 2c_{\sigma^*(j)j} \geq - |R| \delta(A+T)\]
    Vamos juntar essa última desigualdade para todas as instalações não seguras e~\ref{segura} para todas as instalações seguras. Vamos chamar de $P$ o conjunto das instalações seguras,
    \[\sum_{i \in P}\Big( - f_i + \sum_{j:\sigma(j) = i} 2c_{\sigma^*(j)j}\Big) + \sum_{i \not \in P}\Big( \sum_{i^* \in R_i}f_{i^*} - f_i + \sum_{j: \sigma(j)= i } 2c_{\sigma^*(j)j}\Big ) \geq - |R| \delta(A+T)   \]
    É fácil notar que estamos subtraindo o custo de todas as instalações de $S$ e somando $2c_{\sigma^*(j)j}$ para todo cliente $j\in D$. Notemos também que estamos somando o custo de abertura de todas as instalações de $S^*$, uma vez que toda instalação de $S^*$ pertence a exatamente um conjunto $R_i$. Portanto,
    
    \[\sum_{i^* \in S^*}f_{i^*} + 2 \sum_{j \in D} c_{\sigma^*(j)j} - \sum_{i \in S} f_i \geq -m \delta(A+T)\]
    \[A^* + 2 T^* - A \geq - m\delta(A+T)\]
    \[A - A^* - 2T^* \leq m\delta(A+T)\]
\end{proof}
Agora que temos essas duas desigualdades, conseguimos mostrar o seguinte teorema.

\begin{theorem}
    O algoritmo de busca local é uma $3(1+\epsilon)$-aproximação para o problema de localização de instalações.
\end{theorem}
\begin{proof}
    Primeiro, vamos mostrar que o algoritmo roda em tempo polinomial. Claramente, todas as operações podem ser feitas em tempo polinomial, precisamos apenas mostrar que o algoritmo sempre executa um número polinomial de operações. \\
    Seja $M$ o valor da solução inicial. Note que se existir um $k$ tal que $(1-\delta)^kM < 1$, então o algoritmo não fará mais que $k$ operações, uma vez que os custos são inteiros$\red{*}$. Por uma desigualdade conhecida, sabemos que $(1 - \delta)^{\frac{1}{\delta}} \leq \frac{1}{e}$. Quando elevamos tudo por $\ln M$ temos, $(1- \delta)^{\frac{1}{\delta}\ln M} \leq \frac{1}{M}$ e isso nos traz $ (1- \delta)^{\frac{1}{\delta}\ln M}M \leq 1$, como $(1-\delta)$ é estritamente menor que 1, então $\frac{1}{\delta}\ln M + 1$ iterações são suficientes para terminar o programa. Perceba que $\ln M$ é polinomial no tamanho da instância, uma vez que $\ln M$ está apenas a uma constante de $\log M$ e $M$ é no máximo a soma de todos os custos. Como precisamos de $\log$ soma dos custos bits para guardar a instância, então $\log M$ é polinomial no tamanho da instância. \\
    Agora, vamos mostrar a razão de aproximação. Somando as desigualdades encontradas nos Lemas~\ref{lema:3.7} e \ref{lema:3.9}, temos que 
    
        \[A + T - 2A^* - 3T^* \leq 2m\delta(A+T)\] 
        \[(1 - 2m\delta)(A+T) \leq (2A^* + 3T^*) \leq 3 \opt(I)\]
        \[A+T \leq \frac{3}{1-2m\delta}\opt(I)\]
    Queremos que $\frac{1}{1-2m\delta}\leq (1+\epsilon)$.
    \[1 \leq (1+\epsilon)(1-2m\delta)\]
    \[1 - 2m\delta \geq \frac{1}{1+\epsilon}\]
    \[\delta \leq \frac{1}{(2m)} \frac{\epsilon}{(\epsilon+1)}\]
    Assim, podemos escolher $\delta = \frac{\epsilon}{4m}$ e, portanto, teremos um algoritmo com razão de aproximação $3(1+\epsilon)$ que roda com no máximo $\frac{4m}{\epsilon}\ln M$ operações.
\end{proof}
Durante a prova da razão do nosso algoritmo utilizamos que $(2A^* + 3T^*) \leq 3\opt(I)$, vamos utilizar essa folga para melhorar o algoritmo. \\
A partir da instância recebida, vamos criar uma nova instância dividindo o custo de abertura das facilidades por uma constante $\mu \leq 1$. Ao final, vamos encontrar um valor para $\mu$ que irá diminuir a razão de aproximação do algoritmo o máximo possível. \\
Com essa nova instância do algoritmo, vamos encontrar uma solução viável com custo de abertura de instalações $\bar{A}$ e custo de transporte $\bar{T}$. Considere uma solução ótima dessa nova instância com custos $\bar{A}^*$ e $\bar{T}^*$. Para essas soluções, vale
\[ \bar{T} - \bar{A}^* - \bar{T}^* \leq m\delta(\bar{A} + \bar{T})\]
e
\[ \bar{A} - \bar{A}^* - 2\bar{T}^* \leq m\delta(\bar{A} + \bar{T}).\]
Se multiplicarmos os custos de abertura das instalações escolhidas por $\mu$ temos o custo de uma solução viável para a instância original, seja $A = \mu\bar{A}$ e $T$ os custos dessa solução. Vale
\[ T - \bar{A}^* - \bar{T}^* \leq m\delta(\frac{A}{\mu}+ T)\]
e
\[ \frac{A}{\mu} - \bar{A}^* - 2\bar{T}^* \leq m\delta(\frac{A}{\mu} + T),\]
como $\mu \leq 1$ então $m\delta(\frac{A}{\mu}+ T) \leq m\delta\frac{1}{\mu}( A + T)$ e valendo 
\[T - \bar{A}^* - \bar{T}^* \leq m\delta\frac{1}{\mu}( A + T) \] 
e 
\[ \frac{A}{\mu} - \bar{A}^* - 2\bar{T}^* \leq m\delta\frac{1}{\mu}( A + T) .\]
Para uma solução ótima da instância original com custos $A^*$ e $T^*$ conseguimos construir uma solução viável para a instância nova com custos $\frac{A^*}{\mu}$ e $T^*$, essa instância tem custo pelo menos o ótimo, então vale
\[T - \frac{A^*}{\mu} - T^* \leq m\delta\frac{1}{\mu}( A + T) \] 
e 
\[ \frac{A}{\mu} - \frac{A^*}{\mu} - 2T^* \leq m\delta\frac{1}{\mu}( A + T) .\]

Somando a primeira desigualdade com a segunda multiplicada por $\mu$ temos
\[A + T - A^* (1 + \frac{1}{\mu}) - T^*(1 + 2\mu) \leq (1 + \frac{1}{\mu})m\delta(A+T)\]
\[(A+T)(1 - (1+\frac{1}{\mu})m\delta)\leq A^* (1 + \frac{1}{\mu}) + T^*(1 + 2\mu)\]
Note que $(1+\frac{1}{\mu})$ decresce e $1 + 2\mu$ cresce quando $\mu$ cresce. Então, o menor valor do máximo deles dois será quando ele for igual. Igualando eles, encontraremos $\mu = \frac{1}{\sqrt{2}}$ e ambos serão igual a $1 + \sqrt{2}$. Assim temos
\[(A+T)(1 - (1+\frac{1}{\mu})m\delta)\leq A^* (1 + \frac{1}{\mu}) + T^*(1 + 2\mu) \leq (1+\sqrt{2})\opt(I)\]
\[(A+T)\leq \frac{(1+\sqrt{2})}{(1 - (1+\frac{1}{\mu})m\delta)}\opt(I).\]
Fazendo uma análise parecida com a que fizemos para encontrar a razão $3(1 + \epsilon)$, encontramos que escolhendo $\delta = \frac{\epsilon}{2m(1+\sqrt{2})}$ encontramos 
\[(A+T) \leq (1 + \sqrt{2} + \epsilon )\opt(I).\]
Assim, se fizermos a reescala dos custos de abertura das facilidades antes de rodar o algoritmo de busca local temos um novo algoritmo que é uma $(1 + \sqrt{2} + \epsilon )$-aproximação para o nosso problema.
\newpage
\bibliographystyle{plain}
\bibliography{aprox.bib}

\end{document}


