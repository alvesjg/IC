\documentclass[12pt]{article}
\usepackage[margin=3cm]{geometry}
\usepackage{titlesec} % pacote para formatar títulos de seções
\usepackage{tocloft} % pacote para formatar o sumário
\usepackage{hyperref} % pacote para adicionar links no sumário
\usepackage[portuguese]{babel}
\usepackage{setspace}
\usepackage[dvipsnames]{xcolor}
\usepackage{mdframed}
\usepackage{graphicx}
\usepackage{amsmath}
\usepackage{amsthm}
%% Floating package
\usepackage{floatflt,epsfig,epsf}
\renewcommand{\baselinestretch}{1.5}

\newtheorem{theorem}{Teorema}[section]
\newtheorem{corollary}{Corolário}[theorem]
\newtheorem{lemma}[theorem]{Lema} 

\newtheorem{definition}{Definição}[section]
\newtheorem{problem}{Problema}[section] %[defini]
\newcommand{\scs}{\mbox{{\sc scs}}}
\newcommand{\NP}{\mathit{NP}}
\newcommand{\red}[1]{{\color{red}{#1}}}

% \setlength{\parskip}{0.1cm}

\begin{document}
\begin{center}
  
{\Large {\bf Projeto de Pesquisa para Iniciação Científica}

{\large {\em Algoritmos de Aproximação para Problemas de Clustering}} 

%\footnote{ Este pedido est� inserido no Projeto
%     \textsc{Aspectos Estruturais e Algor�tmicos de Objetos
%        Combinat�rios} (Proc.\ FAPESP no.\ 96/04505--2), sob
%      coordena��o do professor Yoshiharu Kohayakawa.  }
}

\vspace{0.2cm}
{\small 
{\bf Orientadora:} Cristina Gomes Fernandes \\
{\bf Aluno:} João Guilherme Alves Santos
}

\vspace{5mm} 

\begin{abstract}
Este é o projeto de pesquisa do aluno de graduação João Guilherme Alves Santos sob supervisão da Profa.\ Dra.\ Cristina Gomes Fernandes. O objetivo desse projeto é estudar e pesquisar algoritmos de aproximação para problemas de clustering. O material estudado fornecerá a João Guilherme o conhecimento necessário para buscar um futuro mestrado na área.
\end{abstract}

\end{center}

\section{Introdução}

Clustering são problemas cujo objetivo é agrupar objetos de maneira que objetos no mesmo cluster apresentam mais semelhanças quando comparados a objetos em clusters diferentes. Tais semelhanças serão definidas pelo problema em questão. Neste projeto, iremos estudar três problemas de clustering: $k$-center, localização de instalações e $k$-median. 


O primeiro deles, o $k$-center, é um problema clássico de otimização combinatória. Dadas $n$ cidades com distâncias especificadas entre elas e um número $k$, o problema quer determinar qual o melhor jeito de construir $k$ depósitos dentre um conjunto determinado das cidades tais que a maior distância entre uma cidade e o depósito mais próximo seja a menor possível.

Nosso problema é modelado como um grafo completo $G(V,E)$, onde $V$ são as cidades e temos uma função distância~$d$ em que $d(e)$ representa a distância das cidades ligadas pela aresta $e$. O objetivo é encontrar um subconjunto $C \subseteq V$ de tamanho $k$ que minimize $\max_{v\in V}d(v,C)$ sendo $d(v,C) = \min_{u\in C}d(vu)$.

Trabalharemos aqui com a versão métrica do problema em que a função $d$ obedece a desigualdade triangular, ou seja, $d(uv) \leq d(ux) + d(xv)$ para todas as triplas de vértices $u$, $x$, $v$. Veremos que, para qualquer função polinomialmente computável~$\alpha(n)$, não existe $\alpha(n)$-aproximação para a versão não métrica desse problema, assumindo $P\not=\NP$. Hsu e Nemhauser~\cite{HSU1979209} mostraram que não existe algoritmo polinomial com razão de aproximação menor que 2 para o problema do $k$-center, assumindo $P\not=\NP$. Assim, temos algoritmos que apresentam o melhor desempenho possível: utilizando o método do gargalo, Gonzalez~\cite{GONZALEZ1985293} e independentemente Hochbaum e Shmoys~\cite{HochShmoys'85} desenvolveram um algoritmo polinomial com razão de aproximação igual a 2. 


Localização de instalações é um problema clássico de otimização que determina a melhor localização para instalações, como fábricas ou depósitos, com base em demandas geográficas, custos de abertura de instalações e distâncias de transporte. Além disso, eles podem ser modelados para outras aplicações como problemas de posicionamento de caches em um computador ou problemas de projeto de redes.

Existem várias versões do problema de localização de instalações. O mais simples deles é o de localização de instalações sem capacidades, ou seja, em que as instalações não tem limitações para suprir clientes.

No problema de localização de instalações sem capacidades, temos um conjunto de clientes $D$ e um conjunto de instalações $F$. Para cada cliente $j \in D$ e cada instalação $i \in F$, há um custo $c_{ij}$ em associar o cliente $j$ à instalação $i$. Além disso, existe um custo de abertura $f_i$ para cada instalação $i \in F$. O objetivo do problema é escolher um subconjunto $F' \subseteq F$ que minimize o custo total de abertura das instalações em~$F'$ somado ao custo de associação de cada cliente $j \in D$ à instalação em~$F'$ mais próxima a ele. Em outras palavras, queremos encontrar $F' \subseteq F$ que minimize $\sum_{i\in F'} f_i + \sum_{j \in D} \min_{i\in F'}c_{ij}$.

Diversos métodos podem ser utilizados para aproximar o problema de localização de instalações. Charikar e Guha desenvolveram um algoritmo com razão de aproximação $2.414$ utilizando o método de busca local~\cite{Charikar&Guha'05}.  Esse problema também pode ser modelado como um problema de programação inteira e, por isso, técnicas envolvendo programação linear podem ser aplicadas a ele.  Por exemplo, há algoritmos que fazem o arredondamento de soluções da relaxação linear do programa inteiro para obter uma solução do problema.  Alguns destes algoritmos atingem boas razões de aproximação, por exemplo, chegando a 1.677~\cite{Byrka&Aardal'10}. Entretanto, a melhor aproximação encontrada utiliza o método primal-dual e garante razão de aproximação 1.488~\cite{LI'13}. Essa não é muito distante do melhor que se poderia encontrar, uma vez que Guha e Khuller mostraram que não existe algoritmo para esse problema com razão de aproximação melhor que 1.463~\cite{GUHA1999228}, a menos que $P = \NP$.

O problema $k$-median é muito parecido com o problema de localização de instalações. A diferença aqui é que não temos custo para a abertura de instalações e podemos abrir no máximo $k$ delas. Assim, como no $k$-center também vamos modelar nosso problema como um grafo e vamos também trabalhar com a versão métrica do problema.

Uma instância do $k$-median consiste em um grafo ($F,C$)-bipartido completo, sendo $F$ o conjunto das instalações e $C$ o conjunto das cidades, um número inteiro $k$ positivo que representa a quantidade de instalações que podem ser abertas e um custo $c_{ij}$ de conexão de cada cidade $j$ a cada instalação $i$. Devemos determinar um conjunto $I \subseteq F$ tal que $|I| \leq k$ de instalações a serem abertas e uma função que atribua cada cidade a uma instalação aberta tal que minimize o custo total de conexão.

Dentre os três problemas apresentados, esse é o que tem a maior folga entre o melhor resultado de inaproximabilidade e a razão do melhor algoritmo de aproximação. Jain, Mahdian e Saberi provaram que não existe algoritmo polinomial com razão de aproximação $1+ \frac{2}{e}$ para o $k$-median ~\cite{JMS'02}, enquanto a melhor aproximação encontrada tem razão $2.675 + \epsilon$ e utiliza o método primal-dual com relaxação Lagrangeana~\cite{BPRST'17}.

\newpage
\section{$k$-Center}

\red{É bom definir o problema novamente?.}\\
\red{Falar sobre cada cluster, centros de cluster, raio. Facilitará nas explicações seguintes.}\\
Vamos definir alguns termos que facilitarão as explicações seguintes. O subconjunto $C \subseteq V$ de tamanho $k$ será formado por vértices que chamaremos de centros de cluster. Os vértices de $V$ serão particionados em $k$ conjuntos chamados clusters e em cada um deles terá exatamente um centro de cluster. Um vértice estará no mesmo cluster que o centro de cluster mais próximo a ele. Cada cluster terá um raio que é a maior distância entre o seu centro e um vértice qualquer dele. O nosso problema se resume a encontrar um conjunto $C$ que minimize o maior desses raios.\\
\red{Mostrar que é NP-difícil.} \\
Antes de começar a falar sobre algoritmos de aproximação para o problema do $k$-center, precisamos mostrar que, assumindo $P\not=\NP$, não existe algoritmo polinomial que resolva nosso problema, ou seja, precisamos mostrar que nosso problema é $\NP$-difícil. Para isso, vamos definir o problema do $k$ conjunto dominante.

\begin{definition}
    Seja $G = (V,E)$ um grafo, um conjunto $D \subseteq V$ é chamado conjunto dominante se para todo vértice $u \in V \setminus D$, existe um vértice $v \in D$ tal que $uv \in E$.
\end{definition}

\begin{problem}[$k$ conjunto dominante]
    Dado um grafo $G$ e um inteiro $k$, decidir se $G$ tem um conjunto dominante $D$ tal que $|D| \leq k$.      
\end{problem}
Esse problema é $\NP$-completo e usaremos ele para mostrar que o problema $k$-center é $\NP$-difícil. \red{Precisa mostrar que ele é $\NP$-completo?}

\begin{theorem}
    O problema $k$-center é $\NP$-difícil.
\end{theorem}

\begin{proof}
    \begin{it}
    Suponha que exista um algoritmo $A$ que resolve o problema do $k$-center em tempo polinomial e seja $I(G,k)$ uma instância do problema $k$ conjunto dominante. Vamos criar uma instância $I'(G',k,d)$ do problema $k$-center reduzindo a instância I. A instância $I'$ tem grafo $G'(V,E')$ completo tal que: \\
    $d(e) = \begin{cases}
            1, \text{ se } e \in E \\
            2, \text{ caso contrário.} 
            \end{cases}
            \forall e \in E'$\\

    Assim, $A$ retorna uma solução para o problema do $k$-center na instância $I'$. É fácil perceber que o conjunto $C \in V $, dos centros de clusters, representa um conjunto dominante em $G'$. Se a solução encontrada pelo algoritmo $A$ tem o maior raio de cluster como 1, quer dizer que todos os vértices então ligados aos centros dos seus cluster com uma aresta que já existia em $G$, assim $C$ representa um conjunto dominante em $G$ e tem tamanho $k$. Como $A$ minimiza esse raio, se a solução encontrada por ele tem maior raio de cluster como 2, não existe uma solução de $I'$ em que os vértices estejam ligados aos centros dos seus clusters apenas por arestas já existentes em $G$ e, por isso, não existe um conjunto dominante de tamanho menor ou igual a $k$ em $G$.

    Portanto, conseguimos resolver em tempo polinomial o problema do $k$ conjunto dominante o que é um absurdo.
        \end{it}
\end{proof}

Agora que justificamos a existência de algoritmos de aproximação para esse problema, podemos começar a falar deles. Começaremos mostrando o porquê de falarmos desses algoritmos apenas para a versão métrica do problema.
\begin{theorem}
    Seja $\alpha(n)$ uma função computável, não existe $\alpha(n)$-aproximação para a versão não métrica do $k$-center.
\end{theorem}

\begin{proof}
    \begin{it}
        A demonstração desse teorema é muito parecida com a do teorema 2.1. \\
        Suponha que exista um algoritmo polinomial $A$ que é uma $\alpha(n)$-aproximação do $k$-center e seja $I(G,k)$ uma instância do problema $k$ conjunto dominante. Vamos criar uma instância $I'(G',k,d)$ do problema $k$-center reduzindo a instância I. A instância $I'$ tem grafo $G'(V,E')$ completo tal que: \\

    $d(e) = \begin{cases}
            1, \text{ se } e \in E \\
            \alpha(n)+1, \text{ caso contrário.} 
            \end{cases}
            \forall e \in E'$\\
    Perceba que $d$ não obedece a desigualdade triangular, uma vez que sabemos que $\alpha(n) > 1$, assumindo $P \not= \NP$.  \\
    Assim, $A$ retorna uma solução viável para o problema do $k$-center na instância $I'$ tal que $\frac{m(A(I'))}{m^*(I')} \leq \alpha(n)$ e, portanto, $m(A(I')) \leq \alpha(n) \; m^*(I')$. \red{definir a função m junto com as outras definições no começo}.
    Como as arestas tem distância 1 ou $\alpha(n)+1$, então existem duas possibilidades de resposta tanto para $m(A(I'))$ quanto para $m^*(I')$ : $1$ ou $\alpha(n) + 1$. Se $m^*(I') = 1$, o único resultado que deixa a inequação verdadeira é $m(A(I'))=1$, se $m^*(I')=\alpha(n)+1$ não existe outro valor possível se não $m(A(I'))=m^*(I')$. \\
    Portanto, nosso algoritmo sempre escolhe o ótimo. Podemos afirmar que se $m^*(I') = 1$ então existe conjunto dominante em $G$ com tamanho menor ou igual a k e, caso contrário, não existe. \\
    Assim, conseguimos resolvemos o $k$ conjunto dominante em tempo polinomial o que, assumindo $P \not= \NP$ é um absurdo.
    \end{it}
\end{proof}
    Fica, então, explicita a impossibilidade de encontrarmos algoritmos de aproximação para a versão não métrica do problema. 
    
\subsection{Algoritmo guloso.}
    
\newpage
\bibliographystyle{plain}
\bibliography{aprox.bib}

\end{document}


