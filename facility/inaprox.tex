Nesta seção vamos mostrar um resultado que, sob certa hipótese, limita inferiormente a razão de aproximação para os algoritmos do problema de localização de instalações métrico. Esse resultado é o Corolário 16.16 que se encontra na Seção 16.2 do livro WS2011. Ele foi provado por Guha e Kuller~\cite{GUHA1999228}.

Para esse resultado vamos precisar definir a versão de otimização do problema de cobertura de conjuntos.
\begin{problem}\label{otsetcover}
    Dada um conjunto $E$ e uma família ${\mathcal{S} = \set{S_1,\ldots,S_m}}$ de subconjuntos de $E$, encontrar $C \subseteq \mathcal{S}$ tal que $\bigcup_{S \in C} S = E$ que minimize $|C|$.
\end{problem}

Esse problema é $\NP$-difícil, uma vez que sua versão de decisão é um dos 21 problemas $\NP$-completos de Karp~\cite{Karp1972}. Além disso, Feige~\cite{Feige98} provou que não existe $(1-\eps)$-aproximação para esse problema, a menos que, para todo problema em $\NP$, exista um algoritmo determinístico que resolva o problema em tempo $n^{O(\log\log n)}$. Assim como ${P = \NP}$ essa também é uma conjectura conhecida na área de complexidade computacional que acreditam ser falsa.

Vamos mostrar que se existe uma $\alpha$-aproximação para o problema de localização de instalações com $\alpha < 1.463$ então existe um algoritmo que é uma $(c \ln n)$-aproximação para o problema da cobertura de conjuntos sem peso com $c<1$.

Vamos apresentar um algoritmo de aproximação para o problema da cobertura de conjuntos que utiliza várias chamadas ao algoritmo que é uma $\alpha$-aproximação para o problema de localização de instalações métrico. 

\begin{algorithm}
    \caption{\sc Inaprox-GK($E,\{S_1,\ldots,S_m\}$)}
    \begin{algorithmic}[1]
        \State $\gamma \gets 0.463$
        \For{$i \gets 1$ até $m$}
            \For{todo $j \in E$}
            \If{$j \in S_i$}
            \State $c_{ij} \gets 1$
            \Else
            \State $c_{ij} \gets 3$
            \EndIf
            \EndFor
        \EndFor
        \For{$k \gets 1$ até $m$}
            \State $I_k \gets \emptyset$
            \State $D \gets E$; \quad $F \gets [m]$
            \While{$D\neq \emptyset$}
                \State $f_i \gets \gamma |D|/k$ para todo $i \in F$
                \State $F' \gets$ {\sc $\alpha$-aprox-loc-inst}$(F,D,c,f)$
                \State $I_k \gets I_k \cup F'$
                \State $D' \gets \{j \in D: c(j,F') = 1 \}$
                \State $F \gets F \setminus F'$;\quad $D\gets D\setminus D'$
            \EndWhile
        \EndFor
    \State \Return $I_k$ que minimize $|I_k|$.
    \end{algorithmic}
\end{algorithm}

Seja $I = (E,\{S_1,\ldots,S_m\})$ uma instância do problema de cobertura de conjuntos. Criaremos uma sequência de instâncias para o problema de localização de instalações métrico.
Para um $k$ fixo, o algoritmo cria uma sequência de instâncias métricas para o problema de localização de instalações. Todas elas terão a mesma função de custo $c$ tal que, para todo $i \in [m]$ e $j \in E$, $c_{ij} = 1$ se $j \in S_i$ e $c_{ij}=3$, caso contrário. É fácil notar que essa função é métrica.
A primeira instância da sequência terá o conjunto de instalações $F \coloneqq [m]$ e o conjunto de clientes $D \coloneqq E$. Além disso, toda instalação terá custo de abertura $f_i \coloneqq \gamma |D|/k$. A ideia é que usamos o algoritmo {\sc $\alpha$-aprox-loc-inst} para encontrar um conjunto $F' \subseteq F$ de instalações que serão abertas e existirá um conjunto $D' \subseteq D$ tal que, para todo $j \in D'$ existe $i \in F'$ tal que $j \in S_i$.
Então, para construir a próxima instância, usaremos o conjunto de instalações $F \setminus F'$ e o conjunto de clientes $D\setminus D'$, que representam os subconjuntos de $E$ que ainda podem ser escolhidos e os elementos de $E$ que ainda não foram cobertos, respectivamente.
Isso acontecerá sucessivamente até que todos os elementos de $E$ sejam cobertos, ou seja, até que $D = \emptyset$.

\begin{theorem}
    \label{inaprox}
Se existe uma $\alpha$-aproximação para o problema da localização de instalações com $\alpha < 1.463$, então {\sc Inaprox-GK} é uma $(c \ln n)$-aproximação para o problema de cobertura de conjuntos com $c<1$ quando $n$ é suficiente grande.
\end{theorem}

\begin{proof}
Seja $k^*$ o tamanho de uma cobertura de conjuntos ótima para a instância $(E,\set{S_1,\ldots,S_m})$ do problema de cobertura de conjuntos. Suponha que estamos na iteração $k^*$ do laço {\bf Para} e na iteração $\ell$ do laço {\bf Enquanto} do algoritmo {\sc Inaprox-GK}.
Sejam $D_\ell$ e $F_\ell$ os conjuntos $D$ e $F$ no começo dessa iteração do laço {\bf Enquanto}. Sejam $n_\ell \coloneqq |D_\ell|$ e $f^\ell \coloneqq \gamma n_\ell /k^*$. Note que existe uma solução para a instância $(D_\ell,F_\ell,c,f^\ell)$ do problema de localização de instalações com custo no máximo $f^\ell k^* + n_\ell$ em que abrimos todas as instalações que representam subconjuntos que estão em uma cobertura de conjuntos ótima.
Seja $F_\ell'$ o conjunto das instalações devolvido pela chamada de {\sc $\alpha$-aprox-loc-inst} na linha~13 e $D_\ell'$ o conjunto de clientes encontrado na linha 15. Sejam $\beta_\ell \coloneqq |F_\ell'|/k^*$ e $p_\ell \coloneqq |D_\ell'|/n_\ell $. É fácil notar que o custo da solução $F'_\ell$ é
\[ |F_\ell'| f^\ell+ p_\ell n_\ell + 3 (1-p_\ell) n_\ell = n_\ell(\beta_\ell \gamma + 3 - 2p_\ell) \leq \alpha ( f^\ell k^* + n_\ell) = \alpha n_\ell (\gamma + 1)\]
e, portanto, 
\begin{equation} 
    \label{inaprox:beta}
\frac{\beta_\ell + 3 - 2p_\ell}{\gamma + 1} \leq \alpha.
\end{equation}

Seja $c$ uma constante entre 0 e 1 que vamos escolher depois. Suponha que $p_\ell \leq 1 - e^{-\frac{\beta_\ell}{c}}$ para algum $\ell$ e $0 < \gamma < 1$. Defina 
\[f(\beta_\ell) \coloneqq \frac{\beta_\ell \gamma + 1 + 2e^{-\frac{\beta_\ell}{c}}}{\gamma + 1}.\]
Note que $f(\beta_\ell) \leq \frac{\beta_\ell + 3 - 2p_\ell}{1 + \gamma} \leq \alpha$.

O valor de $\beta_\ell$ que minimiza $f$ é $c\ln(\frac{2}{\gamma c})$, uma vez que $f'(c\ln(\frac{2}{\gamma c})) = 0$ e $~f''(\beta_\ell)~>~0$ qualquer que seja o valor de $\beta_\ell$. Assim, temos que 
\[f(c\ln\left(\frac{2}{\gamma c}\right)) = \frac{1}{1+\gamma} \left( \gamma c + \ln\left(\frac{2}{\gamma c}\right) + \gamma c + 1 \right).\]
Os valores de $c$ e $\gamma$ que maximizam essa função são $\gamma = 0.463$ e $c$ próximo de 1, e teremos $1.463 \leq f(1.463) \leq \alpha < 1.463$, o que é um absurdo.

Agora, vamos mostrar que, se $p_\ell > 1 - e^{ - \frac{\beta_\ell}{c}}$ para todo $\ell$, então o algoritmo {\sc Inaprox-GK} com $\gamma = 0.463$ devolve uma $(c'\ln n)$-aproximação para o problema da cobertura de conjuntos com $c' < 1$. Vamos chamar de $r$ a quantidade de iterações que o laço {\bf Enquanto} realiza. É evidente que $|I_{k^*}| = k^*\sum_{\ell = 1}^r \beta_\ell$, logo a razão de aproximação desse algoritmo é no máximo $\sum_{\ell = 1}^r \beta_\ell$. Para todo $\ell$, como vale que $p_\ell > 1 - e^{ - \frac{\beta_\ell}{c}}$, então temos que $\beta_\ell < c \ln\left( \frac{1}{1-p_\ell}\right)$. Note que $n_{\ell + 1} = n_\ell(1-p_\ell)$ e $n_r \geq 1$. Assim, vale que $n_1 \prod_{\ell =1}^{r-1} (1 - p_\ell) = n_r \geq 1$ e, consequentemente, $\ln\left(\prod_{\ell =1 }^{r-1} \frac{1}{1-p_\ell}\right) \leq \ln n_1 = \ln n$. Assim,
\[ \sum_{\ell= 1}^r \beta_\ell = \sum_{\ell =1}^{r-1} \beta_\ell + \beta_r < c\sum_{\ell = 1}^{r-1} \ln \left( \frac{1}{1-p\ell}\right) + \beta_r \leq c \ln n + \beta_r. \]
Como $r$ é a última iteração, vale que $p_r= 1$. Então pela desigualdade~\eqref{inaprox:beta} temos que $\beta_r \leq \alpha(1 + \frac{1}{\gamma}) < 4\alpha$ uma vez que escolhemos $\gamma = 0.463$. Portanto,
\[
\sum_{\ell =1}^r \beta_\ell < c \ln n + \beta_r < c\ln n + 4\alpha \leq c' \ln n,
\]
em que a última desigualdade vale para algum $c' \in (c,1)$, uma vez que $4\alpha$ é constante e, pelo enunciado do teorema, podemos assumir que $n$ é suficientemente grande.

Assim, temos uma aproximação para o problema de cobertura de conjuntos com razão de aproximação estritamente menor que $\ln n$ para $n$ suficientemente grande.
\end{proof}

Juntando esse resultado com o resultado do Feige, temos que a existência de uma $\alpha$-aproximação para o problema de localização de instalações métrico com $\alpha < 1.463$ implica a existência de um algoritmo determinístico que leva tempo $O(n^{O(\log \log n)})$ para todo problema $\NP$-completo.