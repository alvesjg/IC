Nessa seção falaremos sobre o algoritmo de busca local para o problema de localização de instalações. Esse algoritmo foi estudado na Seção 9.1 do livro WS2011 e foi proposto primeiramente por Kuen e Hamburger~\cite{KH}. Charikar e Guha~\cite{Charikar&Guha'05} provaram a razão de aproximação igual a 3 e introduziram a ideia de reescala, porém a análise a ser apresentada foi feita por Gupta e Tangwongsan~\cite{DBLP:journals/corr/abs-0809-2554}.

Numa instância $(F,D,c,f)$ do problema de localização de instalações, o custo de transporte está definido apenas para um cliente e uma instalação.
Definimos o custo de transporte entre duas instalações como o menor custo de transporte entre essas duas instalações passando por um cliente qualquer. Então a desigualdade triangular valerá da seguinte forma: para todo $i,j,k \in F \cup D$, 
\[ c_{ij} \leq c_{ik} + c_{kj}.\]

Um algoritmo de busca local começa com uma solução viável para o problema e checa se alguma alteração local melhora o custo da solução atual. Caso essa melhora ocorra, essa alteração é feita. Esse processo se repete até que não exista alteração local que melhore o custo da solução corrente. A solução resultante é chamada de \emph{localmente ótima}. O tempo de execução de uma implementação dessa ideia e a qualidade da solução obtida dependem da definição adotada de alteração local. Quanto mais abrangente essa definição, melhor a solução, porém em geral mais lento será o algoritmo. Quanto mais restrita a definição, mais rápido o algoritmo, porém pior em geral a qualidade da solução final. Usualmente adotam-se restrições mínimas que garantam a polinomialidade do algoritmo.

O algoritmo que descreveremos contará com três operações: abrir instalações fechadas, fechar instalações abertas ou trocar uma instalação aberta por uma fechada. A solução inicial terá todas as instalações abertas. Para garantir a polinomialidade, uma operação só será feita se diminuir o custo da solução atual em uma razão de $1-\delta$, para um $\delta>0$. Como essa solução não é necessariamente localmente ótima, iremos chamá-la de solução \emph{quase localmente ótima}. Ao final, mostraremos que, dado um $\eps > 0$, o algoritmo é uma $3(1 + \epsilon)$-aproximação para o problema de localização de instalações, onde o $\delta$ será escolhido em função do $\epsilon$ e o consumo de tempo do algoritmo será afetado pelo valor de $\delta$.

\begin{algorithm}
    \caption{\sc BuscaLocal$_\eps$-KHCGGT$(F,D,c,f)$}
    \begin{algorithmic}[1]
        \State $\delta \gets \frac{\eps}{(1+\eps)2|F|}$
        \State $ S' \gets F $ 
        \Repeat
        \State $S\gets S'$
        \If{existe $i \in S$ tal que custo$(S\setminus \{i\} ) \leq (1-\delta) $ custo$(S)$}
        \State $S' \gets S \setminus \{i\}$
        \EndIf
        \If{existe $i' \in F\setminus S$ tal que custo$(S \cup \{i'\}) \leq (1-\delta) $ custo$(S)$}
        \State $S' \gets S \cup \{i'\}$
        \EndIf
        \If{existem $i \in S$ e $i' \in F\setminus S$ tal que \\ \hspace{3cm} custo$(S\setminus \{i\} \cup \{i'\}) \leq (1-\delta) $ custo$(S)$}
        \State $S' \gets S \setminus \{i\} \cup \{i'\}$
        \EndIf
        \Until $S=S'$
        \State \Return $S$
    \end{algorithmic}
\end{algorithm}
Vamos mostrar o Lema~\ref{lema:3.12} e o Lema~\ref{lema:3.14} que serão fundamentais para chegar na razão de aproximação do algoritmo. A partir daqui utilizaremos uma instância $I(F,D,c,f)$. Seja $S \subseteq F$ solução devolvida pelo algoritmo {\sc BuscaLocal$_\eps$-KHCGGT$(F,D,c,f)$}. Seja $\sigma : D \rightarrow S$ com $\sigma(j) \coloneqq \arg\min_{i \in S}c_{ij}$. Sejam $A \coloneqq \sum_{i \in S} f_i$ e $T \coloneqq \sum_{j\in D} \min_{i\in S}c_{ij}$. Seja $S^* \subseteq F$ tal que custo$(S^*) =\opt(I)$. Seja $\sigma^* : D \rightarrow S^*$ com $\sigma^*(j) \coloneqq \arg\min_{i \in S^*}c_{ij}$. Sejam também $A^* \coloneqq \sum_{i \in S^*} f_i$ e $T^* \coloneqq \sum_{j\in D} \min_{i\in S^*}c_{ij}$. Seja $m \coloneqq |F|$.
\begin{lemma}
    \label{lema:3.12}
    Sejam $S$ e $S^*$ as soluções como definidas acima. Então, vale que
    \[ T - A^* - T^* \leq m \delta (A+T).\]
\end{lemma}
\begin{proof} 
    Para todo $i^* \in S^* \setminus S$, como a solução $S$ é quase localmente ótima, se abrimos a instalação $i^*$ e associamos a ela em $\sigma$ os clientes que estão associados a ela em $\sigma^*$ não diminuímos o custo da solução de uma fração $1-\delta$. Então,
    \begin{align*} 
      A + f_{i^*} + T + \sum_{j : \sigma^*(j) = i^*} (c_{i^*j} - c_{\sigma(j)j}) &> (1-\delta)(A+T)
    \end{align*} 
que equivale a 
    \begin{align*} 
    f_{i^*} + \sum_{j : \sigma^*(j) = i^*} (c_{i^*j} - c_{\sigma(j)j}) &> -\delta(A+T).
    \end{align*} 
    Vamos agora mostrar que essa desigualdade também é válida para todo $i^* \in S^* \cap S$. Como os clientes sempre estão ligados a uma instalação aberta mais próxima a eles, ao trocar os clientes que estão associados a $i^*$ em $\sigma^*$ para $i^*$ em $\sigma$, o custo de transporte não pode diminuir. Assim, 
    \[ f_{i^*} + \sum_{j : \sigma^*(j) = i^*} (c_{i^*j} - c_{\sigma(j)j}) 
           \ \geq \ \sum_{j : \sigma^*(j) = i^*} (c_{i^*j} - c_{\sigma(j)j})
           \ \geq \ 0 \ \geq \ -\delta(A+T).  \]
    Note que esses dois casos cobrem todas as instalações presentes em $S^*$. Assim, somando as desigualdades para todas essas instalações, vale que
    \begin{subequations}\begin{align*} 
        m\delta(A+T) &\geq  - \sum_{i^* \in S^*}\Big( f_{i^*} + \sum_{j : \sigma^*(j) = i^*} (c_{i^*j} - c_{\sigma(j)j})\Big) \\
        & = -(A^* + \sum_{j \in D} c_{\sigma^*(j)j} - \sum_{j \in D}c_{\sigma(j)j}) \\
        & = - (A^* + T^* - T) = T - A^* - T^*.
    \end{align*}
    \end{subequations}
\end{proof}
Para o Lema~\ref{lema:3.14}, precisaremos de uma função e de um lema que ajudará a limitar o custo da redistribuição de clientes. 
Vamos definir a função ${}\phi : S^* \rightarrow S$ como ${\phi(i^*) = \arg\min_{i \in S} c_{i^*i}}$, ou seja, uma instalação $i$ em $S$ mais próxima a $i^*$ em $S^*$. Então, teremos o seguinte lema.
\begin{lemma}
    \label{lemma:3.8}
    Seja $j$ um cliente tal que $\sigma(j) = i$, $\sigma^*(j) = i^*$, $\phi(i^*) = i'$ e $i\neq i'.$ Então, \[c_{i'j} - c_{ij} \leq 2c_{i^*j}.\]
\end{lemma}
\begin{proof}
    Pela desigualdade triangular, temos que
    \[c_{i'j} \leq c_{i'i^*} + c_{i^*j}.\] Além disso, devido à definição de $i'$, temos que $c_{i'i^*} \leq c_{ii^*}$. Assim,
    \[c_{i'j} \leq c_{ii^*} + c_{i^*j}.\]
    Novamente, pela desigualdade triangular, vale que $c_{ii^*} \leq c_{ij} + c_{i^*j}$. Então,
    \[
        c_{i'j} \leq c_{ij} + 2 c_{i^*j}.
    \] Equivalentemente \[
        c_{i'j} - c_{ij} \leq 2 c_{i^*j}. \]
\end{proof}

\begin{lemma}
    \label{lema:3.14}
    Sejam $S$ e $S^*$ soluções como já definidas. Então,
    \[A - A^* - 2T^* \leq m \delta(A+T).\]
\end{lemma}
\begin{proof}
    Uma instalação $i \in S$ é \emph{segura} se não existe $i^* \in S^*$ tal que $\phi(i^*)=i$.

    Seja $i \in S$ uma instalação segura. Como $S$ é uma solução quase localmente ótima, então se fecharmos $i$ e redistribuirmos cada cliente $j$ que estava associado à $i$ para $\phi(\sigma^*(j))$  não melhoramos o custo da solução em uma razão de $1-\delta$. Então,
    \[
        A - f_i + T + \sum_{j:\sigma(j) = i} (c_{\phi(\sigma^*(j))j} - c_{\sigma(j)j}) > (1-\delta)(A+T),\]
        o que equivale a 
        \[
        - f_i + \sum_{j:\sigma(j) = i} (c_{\phi(\sigma^*(j))j} - c_{\sigma(j)j}) > -\delta(A+T).
        \]
    Como $i$ é uma instalação segura, o Lema~\ref{lemma:3.8} vale para todos os clientes que estão associados a $i$ em $\sigma$. Então
    \begin{align} 
        \label{segura}
        - f_i + \sum_{j:\sigma(j) = i} 2c_{\sigma^*(j)j} &> -\delta(A+T).
    \end{align}

    Seja $i\in S$ uma instalação não segura. Seja $R(i)\coloneqq \{i^* \in S^* : \phi(i^*) = i\}$. Seja $i' \coloneqq \arg\min_{i^* \in R(i)} c_{i^*i}$, ou seja, $i'$ é a instalação de $R(i)$ mais próxima à instalação $i$.
    Para cada $i^* \in R\setminus\{i'\}$, abrir $i^*$ e associar a $i^*$ os clientes que estão associados a $i$ em $\sigma $ e a $i^*$ em $\sigma^*$ não pode melhorar a solução em uma razão de $1-\delta$. Assim
    \[
        A + f_{i^*} + T + \sum_{j: \sigma(j) = i \text{ e } \sigma^*(j) = i^*}(c_{i^*j} - c_{ij}) > (1-\delta)(A+T),\]
    o que equivale a 
        \begin{equation}
        \label{não segura}
        f_{i^*} + \sum_{j: \sigma(j) = i \text{ e } \sigma^*(j) = i^*}(c_{i^*j} - c_{ij}) > -\delta(A+T).        
    \end{equation}    

    Agora vamos ver o que acontece se abrimos $i'$, fecharmos $i$ e associarmos cada cliente $j$ associado a $i$ em $\sigma$ para $\phi(\sigma^*(j))$ se $\sigma^*(j) \not \in R$ e a $i'$ caso contrário. Disso, deduzimos que
    \begin{align*}
        - f_i + f_{i'} + \sum_{j: \sigma(j) = i \text{ e } \sigma^*(j)\not \in R(i)}(c_{\phi(\sigma^*(j))j} - c_{ij}) + \sum_{j: \sigma(j)=i \text{ e }\sigma^*(j) \in R(i)}(c_{i'j} - c_{ij}) &> -\delta(A+T).
    \end{align*}
    No somatório dos clientes $j$ em que $\sigma^*(j) \not \in R(i)$, podemos utilizar o Lema~\ref{lemma:3.8} e obtemos
    \begin{align*}
        - f_i + f_{i'} + \sum_{j: \sigma(j) = i \text{ e } \sigma^*(j)\not \in R} 2c_{\sigma^*(j)j} + \sum_{j: \sigma(j)=i \text{ e }\sigma^*(j) \in R(i)}(c_{i'j} - c_{ij}) &> -\delta(A+T).
    \end{align*}
    Juntando essa desigualdade com~\eqref{não segura} para todas as instalações em $R(i)\setminus\{i'\}$, temos
    \begin{align*}
        -f_i + \sum_{i^* \in R(i)}f_{i^*} + \sum_{j: \sigma(j) = i \text{ e } \sigma^*(j)\not \in R(i)} 2c_{\sigma^*(j)j} + \sum_{j: \sigma(j)=i \text{ e }\sigma^*(j) \in R(i)}(c_{i'j} - c_{ij}) \\+ \sum_{j:\sigma(j)=i \text{ e }\sigma^*(j) \in R(i) \setminus\{i'\}}(c_{\sigma^*(j)j} - c_{ij}) > -|R(i)|\delta(A+T).
    \end{align*}
    Vamos reduzir essa expressão. Seja $j$ tal que $\sigma(j)=i$ e $\sigma^*(j) \in R(i)$. 
    Se $\sigma^*(j) = i'$, então $j$ só aparece no terceiro somatório e certamente $c_{i'j} - c_{ij} \leq 2 c_{i'j}$. 
    Se $\sigma^*(j)\neq i'$, então os termos referentes a $j$ no somatório são $c_{i'j} + c_{\sigma^*(j)j} - 2 \,c_{ij}$. Assim, temos que
    \[
            c_{i'j} + c_{\sigma^*(j)j} - 2 \,c_{ij} \leq 
            c_{\sigma^*(j)j} + c_{i'i} - c_{ij} \leq
            c_{\sigma^*(j)j} + c_{\sigma^*(j)i} - c_{ij} \leq
            2 c_{\sigma^*(j)j}
    \]
    onde a primeira desigualdade vale pois $c_{i'j} \leq c_{i'i} + c_{ij}$, a segunda desigualdade vale pela escolha de $i'$ e a terceira desigualdade vale pois $c_{\sigma^*(j)i} \leq c_{\sigma^*(j)j} + c_{ij}$.

    % Pela desigualdade triangular temos que $c_{i'j} \leq c_{i'i} + c_{ij},$ assim
    % \[c_{i'j} + c_{\sigma^*(j)j} - 2 \,c_{ij} \leq c_{\sigma^*(j)j} + c_{i'i} - c_{ij} \]
    % pela definição do $i'$ temos que $c_{i'i} \leq  c_{\sigma^*(j)i}$ e pela desigualdade triangular temos que $c_{\sigma^*(j)i} \leq c_{\sigma^*(j)j} + c_{ij}$, então
    % \[ c_{\sigma^*(j)j} + c_{i'i} -  c_{ij} \leq 2 c_{\sigma^*(j)j}\]
    Então, para qualquer $j$ temos que o custo referente a $j$ na soma é no máximo $2c_{\sigma^*(j)j}$. Assim,
    \[\sum_{i^* \in R(i)}f_{i^*} - f_i + \sum_{j: \sigma(j)= i } 2c_{\sigma^*(j)j} > - |R(i)| \delta(A+T).\]
    Vamos juntar essa última desigualdade para todas as instalações não seguras e~\eqref{segura} para todas as instalações seguras. Chamando de $P$ o conjunto das instalações seguras, temos que
    \[\sum_{i \in P}\Big( - f_i + \sum_{j:\sigma(j) = i} 2c_{\sigma^*(j)j}\Big) + \sum_{i \in S\setminus P}\Big( \sum_{i^* \in R(i)}f_{i^*} - f_i + \sum_{j: \sigma(j)= i } 2c_{\sigma^*(j)j}\Big ) > - m \delta(A+T). \]
    É fácil notar que estamos subtraindo o custo de todas as instalações de $S$ e somando $2c_{\sigma^*(j)j}$ para todo cliente $j\in D$. Notemos também que estamos somando o custo de abertura de todas as instalações de $S^*$, uma vez que toda instalação de $S^*$ pertence a exatamente um conjunto $R(i)$. Portanto,
    \begin{subequations}
        \begin{align*}
            m\delta(A+T) &\geq - (\sum_{i^* \in S^*}f_{i^*} + 2 \sum_{j \in D} c_{\sigma^*(j)j} - \sum_{i \in S} f_i)\\
            & = - ( A^* + 2 T^* - A ) = A - A^* - 2T^* .
        \end{align*}
    \end{subequations}
\end{proof}
Agora que temos essas duas desigualdades, conseguimos mostrar o seguinte.

\begin{theorem}
    O algoritmo {\sc BuscaLocal$_\eps$-KHCGGT} é uma $3(1+\epsilon)$-aproximação para o problema de localização de instalações.
\end{theorem}
\begin{proof}
    Primeiro, vamos mostrar que o algoritmo roda em tempo polinomial. Claramente, todas as operações podem ser feitas em tempo polinomial, precisamos apenas mostrar que o algoritmo sempre executa um número polinomial de operações.
    Seja $M$ o valor da solução inicial. Note que se existir um inteiro $k$ tal que $(1-\delta)^kM < 1$, então o algoritmo não fará mais que $k$ operações, uma vez que os custos são inteiros. Por uma desigualdade conhecida, sabemos que $(1 - \delta)^{\frac{1}{\delta}} \leq \frac{1}{e}$. Quando elevamos tudo por $\ln M$ temos $(1- \delta)^{\frac{1}{\delta}\ln M} \leq \frac{1}{M}$ e isso nos traz $ (1- \delta)^{\frac{1}{\delta}\ln M}M \leq 1$, como $(1-\delta)$ é estritamente menor que 1, então $\frac{1}{\delta}\ln M + 1$ iterações são suficientes para terminar o programa. Perceba que $\ln M$ é polinomial no tamanho da instância, uma vez que $\ln M$ está apenas a uma constante de $\log M$ e $M$ é no máximo a soma de todos os custos. Como precisamos de $\log$ soma dos custos bits para guardar a instância, então $\log M$ é polinomial no tamanho da instância.

    Agora, vamos mostrar a razão de aproximação. Somando as desigualdades encontradas nos Lemas~\ref{lema:3.12} e \ref{lema:3.14}, temos que 
        \[A + T - 2A^* - 3T^* \leq 2m\delta(A+T)\] 
        e, assim, podemos concluir que
        \[A+T \leq \frac{2A^* + 3T^*}{1-2m\delta} \leq \frac{3}{1-2m\delta} \ \opt(I) = (1+\eps)3 \ \opt(I)\]
        em que a última igualdade vale uma vez que $\delta = \frac{\eps}{(1+\eps)2m}$.
\end{proof}
Durante a prova da razão do algoritmo, utilizamos que $(2A^* + 3T^*) \leq 3\opt(I)$. Vamos utilizar essa folga para melhorar o algoritmo. 
A partir da instância recebida, vamos criar uma nova instância $(F,D,c,\frac{f}{\mu})$ dividindo o custo de abertura das instalações por uma constante $\mu \leq 1$. Ao final, vamos encontrar um valor para $\mu$ que irá diminuir a razão de aproximação do algoritmo o máximo possível. 

Seja $\bar{S}$ a solução devolvida pelo algoritmo {\sc BuscaLocal$_\eps$-KHCGGT} utilizando a instância $(F,D,c,\frac{f}{\mu})$ como entrada. Seja $\bar{A} \coloneqq \sum_{i \in \bar{S}} \frac{f_i}{\mu}$ e $\bar{T} \coloneqq \sum_{j\in D} \min_{i \in \bar{S}} c_{ij}$. Note que existe uma solução para essa nova instância com custos $\frac{A^*}{\mu}$ e $T^*$. Note também que em todos os nossos lemas não utilizamos  que a solução comparada era ótima, então os lemas valem também quando nossa solução é comparada com a solução anterior. Assim, vale que
\[ \overline{T} - \frac{A^*}{\mu} - T^* \leq m\delta(\bar{A} + \overline{T})\]
e
\[ \bar{A} - \frac{A^*}{\mu} - 2T^* \leq m\delta(\bar{A} + \overline{T}).\]
Se multiplicarmos os custos de abertura das instalações escolhidas por $\mu$ temos o custo de uma solução viável para a instância original. Seja $A \coloneqq \mu\bar{A}$ e $T \coloneqq \bar{T}$ os custos dessa solução. Vale
\[ T - \frac{A^*}{\mu} - T^* \leq m\delta(\frac{A}{\mu}+ T)\]
e
\[ \frac{A}{\mu} - \frac{A^*}{\mu} - 2T^* \leq m\delta(\frac{A}{\mu} + T),\]
como $\mu \leq 1$ então $m\delta(\frac{A}{\mu}+ T) \leq m\delta\frac{1}{\mu}( A + T)$ e valendo 
\[T - \frac{A^*}{\mu} - T^* \leq m\delta\frac{1}{\mu}( A + T) \] 
e 
\[ \frac{A}{\mu} - \frac{A^*}{\mu} - 2T^* \leq m\delta\frac{1}{\mu}( A + T) .\]
Somando a primeira desigualdade com a segunda multiplicada por $\mu$ temos
\[A + T - A^* (1 + \frac{1}{\mu}) - T^*(1 + 2\mu) \leq (1 + \frac{1}{\mu})m\delta(A+T),\]
o que é equivalente a 
\[(A+T)(1 - (1+\frac{1}{\mu})m\delta)\leq A^* (1 + \frac{1}{\mu}) + T^*(1 + 2\mu).\]
Note que $(1+\frac{1}{\mu})$ decresce e $1 + 2\mu$ cresce quando $\mu$ cresce. Então, o menor valor do máximo deles dois será quando eles forem iguais. Igualando eles, encontraremos $\mu = \frac{1}{\sqrt{2}}$ e ambos serão iguais a $1 + \sqrt{2}$. Assim temos
\[(A+T)(1 - (1+\frac{1}{\mu})m\delta)\leq A^* (1 + \frac{1}{\mu}) + T^*(1 + 2\mu) \leq (1+\sqrt{2})\opt(I)\]
e, assim, 
\[(A+T)\leq \frac{(1+\sqrt{2})}{(1 - (1+\sqrt{2 })m\delta)}\opt(I).\]
Analogamente ao que foi feito na análise da razão de aproximação $3(1 + \epsilon)$, escolhendo $\delta = \frac{\epsilon}{2m(1+\sqrt{2})}$ encontramos 
\[(A+T) \leq (1 + \sqrt{2} + \epsilon )\opt(I).\]
Assim, se fizermos a reescala dos custos de abertura das facilidades antes de rodar o algoritmo de busca local temos um novo algoritmo que é uma $(1 + \sqrt{2} + \epsilon )$-aproximação para o problema de localização de instalações.