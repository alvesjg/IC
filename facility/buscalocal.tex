Embora o custo de transporte esteja definido apenas para um cliente e uma instalação, aqui falaremos também sobre custos de transporte entre duas instalações. Podemos definir o custo de transporte entre duas instalações como o menor custo de transporte entre essas duas instalações e um cliente qualquer. Então a desigualdade triangular valerá da seguinte forma: para todo $i,j,k \in F \cup D$ 
\[ c_{ij} \leq c_{ik} + c_{kj}.\]

Um algoritmo de busca local começa com uma solução viável para o problema e checa se alguma alteração local melhora o custo da solução atual. Caso essa melhora ocorra, essa alteração é feita. Esse processo se repete até que não exista alteração que melhore o custo e, então, encontramos uma solução que é chamada de \emph{localmente ótima}. A implementação direta da ideia da busca local na maioria das vezes não garante execução em tempo polinomial e, nesses casos, as alterações só acontecerão se satisfizerem alguma restrição. 

Nosso algoritmo contará com três operações: abrir instalações fechadas, fechar instalações abertas ou trocar uma instalação aberta por uma fechada. Nossa solução inicial terá todas as instalações abertas e um cliente sempre estará ligado à instalação aberta mais próxima a ele. Uma operação só será feita se melhorar o custo da solução atual em uma razão de $1-\delta$, para um $\delta>0$. Como essa solução não é necessariamente localmente ótima, iremos chamá-la de solução \emph{quase localmente ótima}. Ao final, mostraremos que, dado um $\eps > 0$, o algoritmo é uma $3(1 + \epsilon)$-aproximação para o problema de localização de instalações, onde o $\delta$ será escolhido em função do $\epsilon$.

Vamos mostrar duas desigualdades que serão fundamentais para chegar na razão de aproximação desejada. A partir daqui utilizaremos uma instância $I$, uma solução quase localmente ótima $(S,\sigma)$ com custo das instalações abertas $A$ e custo de transporte total $T$. Também vamos utilizar uma solução ótima $(S^*,\sigma^*)$ com custo das instalações abertas $A^*$ e custo de transporte total $T^*$. Claramente, $\opt(I) = A^* + T^*$. Também definimos o número total de instalações como $m$.
\begin{lemma}
    \label{lema:3.7}
    Sejam $(S,\sigma)$ e $(S^*,\sigma^*)$ as soluções como definidas acima. Então,
    \[ T - A^* - T^* \leq m \delta (A+T).\]
\end{lemma}
\begin{proof}
    E se $S^* \subseteq S$? 
    Seja $i^* \in S^* \setminus S$, caso exista. Como a solução $(S,\sigma)$ é quase localmente ótima, se abrimos $i^*$ e associamos a ele em $\sigma$ os clientes que estão associados a ele em $\sigma^*$ não diminuímos o custo da solução de uma fração menor que $1-\delta$. Então,
    \begin{align*} 
      A + f_{i^*} + T + \sum_{j : \sigma^*(j) = i^*} (c_{i^*j} - c_{\sigma(j)j}) &\geq (1-\delta)(A+T)
    \end{align*} 
que equivale a 
    \begin{align*} 
    f_{i^*} + \sum_{j : \sigma^*(j) = i^*} (c_{i^*j} - c_{\sigma(j)j}) &\geq -\delta(A+T).
    \end{align*} 
    Vamos agora mostrar que essa desigualdade também é válida para $i^* \in S^* \cap S$, caso exista. Como os clientes sempre estão ligados a uma instalação aberta mais próxima a eles, então ao trocar os clientes que estão associados a $i^*$ em $\sigma^*$ para $i^*$ em $\sigma$ o custo de transporte não pode diminuir. Assim, 
    \[
        f_{i^*} + \sum_{j : \sigma^*(j) = i^*} (c_{i^*j} - c_{\sigma(j)j}) 
           \ \geq \ \sum_{j : \sigma^*(j) = i^*} (c_{i^*j} - c_{\sigma(j)j})
           \ \geq \ 0 \ \geq \ -\delta(A+T).
    \]

    Note que esses dois casos cobrem todas as instalações presentes em $S^*$. Assim, somando as desigualdades para todas essas instalações sabemos que o seguinte vale:
    \begin{subequations}\begin{align*} 
        m\delta(A+T) &\geq  - \sum_{i^* \in S^*}\Big( f_{i^*} + \sum_{j : \sigma^*(j) = i^*} (c_{i^*j} - c_{\sigma(j)j})\Big) \\
        & = -(A^* + \sum_{j \in D} (c_{\sigma^*(j)j}) - \sum_{j \in D}(c_{\sigma(j)j})) \\
        & = - (A^* + T^* - T) = T - A^* - T^*.
    \end{align*}
    \end{subequations}
\end{proof}
Para a segunda desigualdade, precisaremos de uma função e de um lema que ajudará a limitar o custo da redesignação de clientes. 
Vamos definir a função $\phi : S^* \rightarrow S$ como \[\phi(i^*) = \arg\min_{i \in S} c_{i^*i},\] ou seja, a instalação $i$ em $S$ mais próxima a $i^*$ em $S^*$. Então, teremos o seguinte lema
\begin{lemma}
    \label{lemma:3.8}
    Seja $j$ um cliente tal que $\sigma(j) = i$, $\sigma^*(j) = i^*$, $\phi(i^*) = i'$ e $i\neq i'.$ Então, \[c_{i'j} - c_{ij} \leq 2c_{i^*j}.\]
\end{lemma}

\begin{proof}
    Pela desigualdade triangular temos que
    \[c_{i'j} \leq c_{i'i^*} + c_{i^*j},\] além disso, devido a definição de $i'$, temos que $c_{i'i^*} \leq c_{ii^*}$. Assim,
    \[c_{i'j} \leq c_{ii^*} + c_{i^*j}.\]
    Novamente, pela desigualdade triangular, vale que $c_{ii^*} \leq c_{ij} + c_{i^*j}$. Então,
    \[
        c_{i'j} \leq c_{ij} + 2 c_{i^*j} 
    \] equivalentemente \[
        c_{i'j} - c_{ij} \leq 2 c_{i^*j}. \]
\end{proof}

Agora, conseguimos provar a segunda desigualdade que será necessária para a razão de aproximação.
\begin{lemma}
    \label{lema:3.9}
    Seja $(S,\sigma)$ e $(S^*,\sigma^*)$ soluções como já definidas. Então,
    \[A - A^* - 2T^* \leq m \delta(A+T).\]
\end{lemma}
\begin{proof}
    Antes de começar a demonstração em si, vamos definir o que é uma instalação segura. Uma instalação $i \in S$ é \emph{segura} se não existe $i^* \in S^*$ tal que $\phi(i^*)=i$.

    Seja $i \in S$ uma instalação segura. Como $(S,\sigma)$ é uma solução parcialmente localmente ótima, então se fecharmos $i$ e associarmos cada cliente j associado a $i$ para $\phi(\sigma^*(j))$ não melhoraremos o custo da solução em uma razão menor que $1-\delta$. Então,
    \[
        A - f_i + T + \sum_{j:\sigma(j) = i} (c_{\phi(\sigma^*(j))j} - c_{\sigma(j)j}) \geq (1-\delta)(A+T)\]
        o que equivale a 
        \[
        - f_i + \sum_{j:\sigma(j) = i} (c_{\phi(\sigma^*(j))j} - c_{\sigma(j)j}) \geq -\delta(A+T).
        \]
    Como $i$ é uma instalação segura o Lema~\ref{lemma:3.8} vale para todos os clientes que estão associados a $i$ em $\sigma$, então
    \begin{align} 
        \label{segura}
        - f_i + \sum_{j:\sigma(j) = i} 2c_{\sigma^*(j)j} &\geq -\delta(A+T).
    \end{align}

    Seja $i\in S$ uma instalação não segura. Vamos definir o conjunto $R \subseteq S^*$ como o conjunto das instalações que fazem com que $i$ não seja segura, ou seja, $R = \{i^* \in S^* : \phi(i^*) = i\}$. Vamos definir também $i' = \arg\min_{i^* \in R} c_{i^*i}$, ou seja, a instalação de $R$ mais próxima à instalação $i$.
    Seja $i^* \in R\setminus\{i'\}$, caso esse conjunto não seja vazio. Abrir $i^*$ e associar a $i^*$ os clientes que estão associados a $i$ em $\sigma $ e a $i^*$ em $\sigma^*$ não pode melhorar a resposta em uma razão melhor que $(1-\delta)$, assim
    \[
        A + f_i + T + \sum_{j: \sigma(j) = i \text{ \& } \sigma^*(j) = i^*}(c_{i^*j} - c_{ij})\geq (1-\delta)(A+T)\]
    o que equivale a 
        \begin{equation}
        \label{não segura}
        f_i + \sum_{j: \sigma(j) = i \text{ \& } \sigma^*(j) = i^*}(c_{i^*j} - c_{ij}) \geq -\delta(A+T).        
    \end{equation}    

    Agora vamos ver o que acontece se abrimos $i'$, fecharmos $i$ e associarmos os clientes $j$ associados a $i$ em $\sigma$ para $\phi(\sigma^*(j))$ se $\sigma^*(j) \not \in R$ e a $i'$ caso contrário
    \begin{align*}
        - f_i + f_{i'} + \sum_{j: \sigma(j) = i \text{ \& } \sigma^*(j)\not \in R}(c_{\phi(\sigma^*(j)j)} - c_{ij}) + \sum_{j: \sigma(j)=i \text{ \& }\sigma^*(j) \in R}(c_{i'j} - c_{ij}) &\geq -\delta(A+T).
    \end{align*}
    No somatório dos clientes $j$ em que $\sigma^*(j) \not \in R$ podemos utilizar o Lema~\ref{lemma:3.8},
    \begin{align*}
        - f_i + f_{i'} + \sum_{j: \sigma(j) = i \text{ \& } \sigma^*(j)\not \in R} 2c_{\sigma^*(j)j} + \sum_{j: \sigma(j)=i \text{ \& }\sigma^*(j) \in R}(c_{i'j} - c_{ij}) &\geq -\delta(A+T)
    \end{align*}
    juntando essa desigualdade com \ref{não segura} para todas as instalações em $R\setminus\{i'\}$, temos
    \begin{align*}
        -f_i + \sum_{i^* \in R}f_{i^*} + \sum_{j: \sigma(j) = i \text{ \& } \sigma^*(j)\not \in R} 2c_{\sigma^*(j)j} + \sum_{j: \sigma(j)=i \text{ \& }\sigma^*(j) \in R}(c_{i'j} - c_{ij}) \\+ \sum_{j:\sigma(j)=i \text{ \& }\sigma^*(j) \in R \setminus\{i'\}}(c_{\sigma^*(j)j} - c_{ij}) \geq -|R|\delta(A+T).
    \end{align*}
    Vamos reduzir essa expressão. Seja $j$ tal que $\sigma(j)=i$ e $\sigma^*(j) \in R$. 
    Se $\sigma^*(j) = i'$, então $j$ só aparece no terceiro somatório e certamente $c_{i'j} - c_{ij} \leq 2 c_{i'j}$. 
    Se $\sigma^*(j)\neq i'$, então na soma temos $c_{i'j} + c_{\sigma^*(j)j} - 2 \,c_{ij}$. Assim, temos que
    \[
            c_{i'j} + c_{\sigma^*(j)j} - 2 \,c_{ij} \leq 
            c_{\sigma^*(j)j} + c_{i'i} - c_{ij} \leq
            c_{\sigma^*(j)j} + c_{\sigma^*(j)i} - c_{ij} \leq
            2 c_{\sigma^*(j)j}
    \]
    onde a primeira desigualdade vale pois $c_{i'j} \leq c_{i'i} + c_{ij}$, a segunda desigualdade vale pela escolha de $i'$ e a terceira desigualdade vale pois $c_{\sigma^*(j)i} \leq c_{\sigma^*(j)j} + c_{ij}$.

    % Pela desigualdade triangular temos que $c_{i'j} \leq c_{i'i} + c_{ij},$ assim
    % \[c_{i'j} + c_{\sigma^*(j)j} - 2 \,c_{ij} \leq c_{\sigma^*(j)j} + c_{i'i} - c_{ij} \]
    % pela definição do $i'$ temos que $c_{i'i} \leq  c_{\sigma^*(j)i}$ e pela desigualdade triangular temos que $c_{\sigma^*(j)i} \leq c_{\sigma^*(j)j} + c_{ij}$, então
    % \[ c_{\sigma^*(j)j} + c_{i'i} -  c_{ij} \leq 2 c_{\sigma^*(j)j}\]
    Então, para qualquer $j$ temos que o custo referente a $j$ na soma é no máximo $2c_{\sigma^*(j)j}$. Assim,
    \[\sum_{i^* \in R}f_{i^*} - f_i + \sum_{j: \sigma(j)= i } 2c_{\sigma^*(j)j} \geq - |R| \delta(A+T).\]
    Vamos juntar essa última desigualdade para todas as instalações não seguras e~\eqref{segura} para todas as instalações seguras. Chamando de $P$ o conjunto das instalações seguras, temos que
    \[\sum_{i \in P}\Big( - f_i + \sum_{j:\sigma(j) = i} 2c_{\sigma^*(j)j}\Big) + \sum_{i \not \in P}\Big( \sum_{i^* \in R_i}f_{i^*} - f_i + \sum_{j: \sigma(j)= i } 2c_{\sigma^*(j)j}\Big ) \geq - |R| \delta(A+T). \]
    É fácil notar que estamos subtraindo o custo de todas as instalações de $S$ e somando $2c_{\sigma^*(j)j}$ para todo cliente $j\in D$. Notemos também que estamos somando o custo de abertura de todas as instalações de $S^*$, uma vez que toda instalação de $S^*$ pertence a exatamente um conjunto $R_i$. Portanto,
    \begin{subequations}
        \begin{align*}
            m\delta(A+T) &\geq - (\sum_{i^* \in S^*}f_{i^*} + 2 \sum_{j \in D} c_{\sigma^*(j)j} - \sum_{i \in S} f_i)\\
            & = - ( A^* + 2 T^* - A ) = A - A^* - 2T^* .
        \end{align*}
    \end{subequations}
\end{proof}
Agora que temos essas duas desigualdades, conseguimos mostrar o seguinte teorema.

\begin{theorem}
    O algoritmo de busca local é uma $3(1+\epsilon)$-aproximação para o problema de localização de instalações.
\end{theorem}
\begin{proof}
    Primeiro, vamos mostrar que o algoritmo roda em tempo polinomial. Claramente, todas as operações podem ser feitas em tempo polinomial, precisamos apenas mostrar que o algoritmo sempre executa um número polinomial de operações.
    Seja $M$ o valor da solução inicial. Note que se existir um inteiro $k$ tal que $(1-\delta)^kM < 1$, então o algoritmo não fará mais que $k$ operações, uma vez que os custos são inteiros$\red{*}$. Por uma desigualdade conhecida, sabemos que $(1 - \delta)^{\frac{1}{\delta}} \leq \frac{1}{e}$. Quando elevamos tudo por $\ln M$ temos $(1- \delta)^{\frac{1}{\delta}\ln M} \leq \frac{1}{M}$ e isso nos traz $ (1- \delta)^{\frac{1}{\delta}\ln M}M \leq 1$, como $(1-\delta)$ é estritamente menor que 1, então $\frac{1}{\delta}\ln M + 1$ iterações são suficientes para terminar o programa. Perceba que $\ln M$ é polinomial no tamanho da instância, uma vez que $\ln M$ está apenas a uma constante de $\log M$ e $M$ é no máximo a soma de todos os custos. Como precisamos de $\log$ soma dos custos bits para guardar a instância, então $\log M$ é polinomial no tamanho da instância.

    Agora, vamos mostrar a razão de aproximação. Somando as desigualdades encontradas nos Lemas~\ref{lema:3.7} e \ref{lema:3.9}, temos que 
        \[A + T - 2A^* - 3T^* \leq 2m\delta(A+T)\] 
        e, assim, podemos concluir que
        \[A+T \leq \frac{2A^* + 3T^*}{1-2m\delta} \leq \frac{3}{1-2m\delta} \ \opt(I).\]
    Para chegar onde queremos precisamos garantir que \[\frac{1}{1-2m\delta}\leq (1+\epsilon)\]
    o que é equivalente a 
    %\[1 \leq (1+\epsilon)(1-2m\delta)\]
    %\[1 - 2m\delta \geq \frac{1}{1+\epsilon}\]
    \[\delta \leq \frac{1}{(2m)} \frac{\epsilon}{(\epsilon+1)}\]
    caso $ 1 - 2m\delta >0$.
    Assim, podemos escolher $\delta = \frac{\epsilon}{4m}$ e, portanto, teremos um algoritmo com razão de aproximação $3(1+\epsilon)$ que roda com no máximo $\frac{4m}{\epsilon}\ln M$ operações.
\end{proof}
Durante a prova da razão do algoritmo, utilizamos que $(2A^* + 3T^*) \leq 3\opt(I)$. Vamos utilizar essa folga para melhorar o algoritmo. 
A partir da instância recebida, vamos criar uma nova instância dividindo o custo de abertura das facilidades por uma constante $\mu \leq 1$. Ao final, vamos encontrar um valor para $\mu$ que irá diminuir a razão de aproximação do algoritmo o máximo possível. 

Com essa nova instância do algoritmo, vamos encontrar uma solução viável com custo de abertura de instalações $\bar{A}$ e custo de transporte $\overline{T}$. Note que existe uma solução para essa nova instância com custos $\frac{A^*}{\mu}$ e $T^*$. Note também que em todos os nossos lemas não utilizamos  que a solução comparada era ótima, então os lemas valem também quando nossa solução é comparada com a solução anterior. Assim, vale que
\[ \overline{T} - \frac{A^*}{\mu} - T^* \leq m\delta(\bar{A} + \overline{T})\]
e
\[ \bar{A} - \frac{A^*}{\mu} - 2T^* \leq m\delta(\bar{A} + \overline{T}).\]
Se multiplicarmos os custos de abertura das instalações escolhidas por $\mu$ temos o custo de uma solução viável para a instância original, seja $A = \mu\bar{A}$ e $T$ os custos dessa solução. Vale
\[ T - \frac{A^*}{\mu} - T^* \leq m\delta(\frac{A}{\mu}+ T)\]
e
\[ \frac{A}{\mu} - \frac{A^*}{\mu} - 2T^* \leq m\delta(\frac{A}{\mu} + T),\]
como $\mu \leq 1$ então $m\delta(\frac{A}{\mu}+ T) \leq m\delta\frac{1}{\mu}( A + T)$ e valendo 
\[T - \frac{A^*}{\mu} - T^* \leq m\delta\frac{1}{\mu}( A + T) \] 
e 
\[ \frac{A}{\mu} - \frac{A^*}{\mu} - 2T^* \leq m\delta\frac{1}{\mu}( A + T) .\]
Somando a primeira desigualdade com a segunda multiplicada por $\mu$ temos
\[A + T - A^* (1 + \frac{1}{\mu}) - T^*(1 + 2\mu) \leq (1 + \frac{1}{\mu})m\delta(A+T)\]
o que é equivalente a 
\[(A+T)(1 - (1+\frac{1}{\mu})m\delta)\leq A^* (1 + \frac{1}{\mu}) + T^*(1 + 2\mu).\]
Note que $(1+\frac{1}{\mu})$ decresce e $1 + 2\mu$ cresce quando $\mu$ cresce. Então, o menor valor do máximo deles dois será quando eles forem iguais. Igualando eles, encontraremos $\mu = \frac{1}{\sqrt{2}}$ e ambos serão iguais a $1 + \sqrt{2}$. Assim temos
\[(A+T)(1 - (1+\frac{1}{\mu})m\delta)\leq A^* (1 + \frac{1}{\mu}) + T^*(1 + 2\mu) \leq (1+\sqrt{2})\opt(I)\]
e, assim, 
\[(A+T)\leq \frac{(1+\sqrt{2})}{(1 - (1+\sqrt{2 })m\delta)}\opt(I).\]
Analogamente ao que foi feito na análise da razão de aproximação $3(1 + \epsilon)$, escolhendo $\delta = \frac{\epsilon}{2m(1+\sqrt{2})}$ encontramos 
\[(A+T) \leq (1 + \sqrt{2} + \epsilon )\opt(I).\]
Assim, se fizermos a reescala dos custos de abertura das facilidades antes de rodar o algoritmo de busca local temos um novo algoritmo que é uma $(1 + \sqrt{2} + \epsilon )$-aproximação para o nosso problema.

A análise feita aqui é de autoria de Gupta e Tangwongsan~\cite{DBLP:journals/corr/abs-0809-2554}.