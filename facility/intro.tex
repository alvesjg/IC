Antes de falarmos sobre algoritmos de aproximação para o problema da localização de instalações, vamos mostrar que, assumindo $P\neq\NP$, não existe algoritmo polinomial que resolva nosso problema, ou seja, vamos mostrar que nosso problema é $\NP$-difícil. Para isso, vamos definir o problema da cobertura por vértices.



\begin{problem}(Cobertura por vértices)
    Dado um grafo $G$ e um inteiro $k$, decidir se $G$ tem uma cobertura por vértices de tamanho $k$.
\end{problem}
Esse problema é $\NP$-completo, sendo um dos famosos $21$ problemas do Karp~\cite{Karp1972}. Disso deriva-se o seguinte.

\begin{theorem}
    O problema de localização de instalações é $\NP$-difícil.
\end{theorem}

\begin{proof}
    Suponha que exista um algoritmo $A$ que resolva o problema de localização de instalações em tempo polinomial. \\
    Seja $I(G,k)$ uma instância do problema da cobertura por vértices em que $G = (V,E)$. Vamos criar uma instância $I'(F,D,c,f)$ do problema de localização de instalações a partir de $I$.\\
    O conjunto de possíveis instalações a serem abertas $F$ será o conjunto de vértices $V$. O conjunto de clientes $D$ será o conjunto de arestas $E$. O custo de abertura $f_i$ será 1 para cada instalação $i$. O custo de associação $c_{ij} \text{, para } i \in F \text{ e } j \in D$, será 0 se o vértice referente a $i$ é extremo da aresta referente a $j$ e 2 caso contrário. \\
    A resposta ótima será a menor quantidade de instalações que precisamos abrir até que todos os clientes possam ser associados a instalações em que o seu custo de associação é 0. Se um cliente(aresta) está associado a uma instalação(vértice) cujo custo de associação é 2, podemos diminuir o custo total abrindo uma instalação em que o custo de sua associação é 0, uma vez que abrir a instalação custaria 1 e diminuiríamos o custo de associação do cliente de 2 para 0. Sabemos que essa instalação existe, pois para cada cliente(aresta) existem duas instalações com custo de associação 0(seus vértices extremos). \\
    Assim, é fácil notar que essa quantidade de instalações é o tamanho mínimo de uma cobertura por vértices em $G$. Portanto, se o custo total da solução do algoritmo $A$ aplicado à instância $I'$ é menor ou igual a $k$, então a resposta para $I$ é sim. Caso contrário, a resposta é não. \\
    Desse modo, resolvemos o problema da cobertura por vértices em tempo polinomial, o que, assumindo $P\neq\NP$, é um absurdo.
\end{proof}