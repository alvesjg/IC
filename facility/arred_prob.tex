Nessa seção vamos apresentar o algoritmo que faz arredondamento probabilístico de uma solução ótima de~\eqref{P1} e~\eqref{D1}. Esse algoritmo foi estudado na seção 5.8 do livro WS2011 e foi desenvolvido pelo Chudak e Shmoys~\cite{Chudak2003}. Vamos utilizar os programas lineares~\eqref{P1} e~\eqref{D1}.

Relembre a Definição~\ref{def:3.7}. Além disso, para uma solução ótima $(x^*,y^*)$ de~\eqref{P1}, denotamos $C_j^* \coloneqq \sum_{i \in F} x_{ij}^*c_{ij}$ para todo cliente $j \in D$.  


\begin{algorithm}[tbh]
\caption{\sc ArredProb-CS($F,D,c,f$)}
\begin{algorithmic}[1]
        \State Sejam $(x^*,y^*)$ e $(v^*,y^*)$ soluções ótimas de~\eqref{P1} e~\eqref{D1}, respectivamente.
        \State $X \gets \emptyset$
        \State $S \gets D$
        \While{$S \neq \emptyset$}
        \State Escolha $j \in S$ que minimize $v^*_j + C^*_j$
        \State Escolha uma instalação em $N(j)$ aleatoriamente com probabilidade $x^*_{ij}$ para a instalação $i$.
        \State $X \gets X \cup \{i\}$
        \State $S \gets S \setminus N^2(j)$
        \EndWhile
        \State \Return $X$
\end{algorithmic}
\end{algorithm}

\begin{theorem} O algoritmo {\sc ArredProb-CS} é uma 3-aproximação para o problema de localização de instalações métrico.
\end{theorem}

\begin{proof}
É evidente que o algoritmo toma tempo polinomial.

Seja $j_k$ o cliente escolhido na linha 5 de uma iteração $k$ qualquer do algoritmo e $S_k$ o conjunto $S$ no início da iteração $k$. Note que, para um mesmo par de soluções ótimas de~\eqref{P1} e~\eqref{D1}, a escolha de $j_k$ para uma iteração $k$ qualquer é determinística e, portanto, $N^2(j_k) \cap S_k$ é sempre igual para uma iteração $k$. Portanto, a parte probabilística do algoritmo se da apenas na escolha de uma instalação $i \in N(j)$. Seja $i_k$ a variável aleatória que indica a instalação em $N(j_k)$ que será escolhida na linha 6 do algoritmo. Assim, temos que $P(i_k = i) = x_{ij_k}^*$ para todo $i \in N(j_k)$. Desse modo, o custo esperado de abertura para $i_k$ é 

\[ \mathbb{E}[f_{i_k}] = \sum_{i \in N(j_k)} f_i P(i_k = i) = \sum_{i \in N(j_k)} f_i x_{ij_k}^* \leq \sum_{i \in N(j_k)} f_i y^*_i\]
onde a desigualdade vale pela restrição~\ref{P3}. Seja $c_{i_kj}$ a variável aleatória que representa o custo de transporte do cliente $j \in N^2(j_k) \cap S_k$. Para $j_k$ temos que 
\[\mathbb{E}[c_{i_kj_k}] = \sum_{i \in N(j_k)} c_{ij_k} P(i_k = i) = \sum_{i \in N(j_k)} c_{ij_k}x^*_{ij_k} = C^*_{j_k}. \]
Seja $\ell$ um cliente em $N^2(j_k) \cap S_k$ diferente de $j_k$ e $h \in N(j_k)$ tal que $x^*_{h\ell} > 0$. Note que, pela definição de $N^2(j_k)$, $h$ existe. Pela desigualdade triangular vale que $c_{i\ell} \leq c_{ij_k} + c_{hj_k} + c_{hj}$ e, portanto,
\begin{subequations} 
        \begin{align*}
        \mathbb{E}[c_{i_k\ell}] = \sum_{i \in N(j_k)} c_{i\ell} P(i_k = i) &= \sum_{i \in N(j_k)} c_{i\ell}x^*_{ij_k}\\
        &\leq \sum_{i \in N(j_k)} (c_{ij_k} + c_{hj} + c_{h\ell})x^*_{ij_k} \\
        &= c_{hj} + c_{h\ell} + C^*_{j_k}\\
        &\leq v^*_\ell + v^*_{j_k} + C^*_{j_k} \\
        &\leq 2v^*_{\ell} + C^*_\ell 
        \end{align*}
\end{subequations}
onde a segunda desigualdade vale pelo Lema~\ref{lemma:3.5} e a terceira vale pela escolha de $j_k$.

Então fica evidente que o valor esperado da nossa solução é 
\begin{subequations}
        \begin{align*}
                \sum_k {\big (}\mathbb{E}[f_{i_k}] + \sum_{j \in N^2(j_k)\cap S_k}\mathbb{E}[c_{i_kj}] {\big )} &\leq \sum_k {\big (} \sum_{i \in N(j_k)}f_iy^*_i + \sum_{j \in N^2(j_k) \cap S_k} (2v^*_j + C^*_j) {\big )}  \\
                & = \sum_k \sum_{i \in N(j_k)}f_iy^*_i + \sum_k \sum_{j \in N^2(j_k)\cap S_k}(2v^*_j + C^*_j)\\
                &\leq\sum_{i \in F} f_iy^*_i + \sum_{j \in D }(2v^*_j + C^*_j)\\
                &= \sum_{i \in F} f_iy^*_i + \sum_{i \in F, j \in D} c_{ij}x^*_{ij} + 2\sum_{j \in D}v_j \\
                &\leq 3 \opt(I)
        \end{align*}
\end{subequations}
onde a segunda igualdade vale pois para $k_1 < k_2$, vale que $N(j_{k_1}) \cap N(j_{k_2}) = \emptyset$, caso contrário $j_{k_2}$ estaria em $N^2(j_{k_1})$ e, portanto, não estaria em $S_{k_2}$.
\end{proof}

