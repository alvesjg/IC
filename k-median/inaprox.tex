Nesta seção vamos mostrar um resultado que limita inferiormente a razão de aproximação para os algoritmos do problema das $k$-medianas. Esse resultado é o Teorema 16.17 que se encontra na Seção 16.2 do livro WS2011. Ele foi desenvolvido por Jain, Mahdian, Markakis, Saberi e Vazirani~\cite{jain2002greedy}.

Vamos mostrar que se existe uma $\alpha$-aproximação para o problema das $k$-medianas com $\alpha < 1.735$ então existe um algoritmo que é uma $(c \ln n)$-aproximação para o problema da cobertura de conjuntos sem peso com $c<1$. 
Feige~\cite{Feige98} mostrou que a existência desse segundo algoritmo implica a existência de um algoritmo determinístico que leva tempo $O(n^{O(\log\log n)})$ para todo problema $\NP$-completo.

Vamos apresentar um algoritmo de aproximação para o problema da cobertura de conjuntos que utiliza várias chamadas ao algoritmo que é uma $\alpha$-aproximação para o problema das $k$-medianas. 
Seja $(E,\{S_1,\ldots,S_m\})$ uma instância do problema de cobertura de conjuntos. Criaremos uma sequência de instâncias para o problema das $k$-medianas. 
A primeira delas terá o conjunto de instalações $F \coloneqq [m]$ e o conjunto de clientes $D \coloneqq E$. 
A ideia é que escolheremos um conjunto $F' \subseteq F$ de instalações que serão abertas e existirá um conjunto $D' \subseteq D$ tal que, para todo $j \in D'$, existe $i \in F'$ tal que $j \in S_i$. 
Então, para construir a próxima instância, usaremos o conjunto de instalações $F \setminus F'$ e o conjunto de clientes $D \setminus D'$, que representam os subconjuntos de $E$ que ainda podem ser escolhidos e os elementos de $E$ que ainda não foram cobertos, respectivamente.
Isso acontecerá sucessivamente até que todo os elementos de $E$ sejam cobertos, ou seja, até que $D = \emptyset$.
O custo de conexão dos clientes será igual para todas as instâncias. Para uma instalação $i$ e um cliente $j$ definimos $c_{ij}\coloneqq 1$ se $j \in S_i$ e $c_{ij} \coloneqq 3$ caso contrário.
Queremos escolher $k^*$ instalações em cada chamada do algoritmo do problema das $k$-medianas, em que $k^*$ é o menor tamanho de uma cobertura de conjunto para a instância $(E,\set{S_1,\ldots,S_m})$ do problema de cobertura de conjuntos. Como é difícil saber o tamanho de uma cobertura de conjuntos ótima, vamos rodar essa rotina para todo $k \in [m]$, ou seja, para todos os candidatos para o tamanho $k^*$ de uma cobertura de conjuntos ótima para essa instância.

\begin{algorithm}
    \caption{\sc Inaprox-JMMSV$(E,\{S_1,\ldots,S_m\})$}
    \begin{algorithmic}[1]
        \For{$i \gets 1$ até $m$}
            \For{todo $j \in E$}
            \If{$j \in S_i$}
            \State $c_{ij} \gets 1$
            \Else
            \State $c_{ij} \gets 3$
            \EndIf
            \EndFor
        \EndFor
        \For{$k \gets 1$ até $m$}
            \State $I_k \gets \emptyset$
            \State $D \gets E$; $F \gets [m]$
            \While{$D\neq \emptyset$}
                \State $F' \gets$ {\sc $\alpha$-aprox-k-med}$(F,D,c,k)$
                \State $I_k \gets I_k \cup F'$
                \State $D' \gets \{j \in D: c(j,F') = 1 \}$
                \State $F \gets F \setminus F'$; $D\gets D\setminus D'$
            \EndWhile
        \EndFor
    \State \Return $I_k$ que minimize $|I_k|$.
    \end{algorithmic}
\end{algorithm}


\begin{theorem}
Se existe uma $\alpha$-aproximação para o problema das $k$-medianas com $\alpha < 1.735$, então {\sc Inaprox-JMMSV} é uma $(c \ln n)$-aproximação para o problema de cobertura de conjuntos com $c<1$ quando $n$ é suficiente grande.
\end{theorem}

\begin{proof}
Seja $k^*$ o tamanho de uma cobertura de conjuntos ótima da instância $(E,\{S_1,\ldots,S_m\})$ do problema de cobertura de conjuntos. Suponha que estamos na iteração $k^*$ do laço {\bf Para} e na iteração $\ell$ do laço {\bf Enquanto} do algoritmo {\sc Inaprox-JMMSV}.
Seja $D_\ell$ e $F_\ell$ os conjuntos $D$ e $F$ no começo dessa iteração do laço {\bf Enquanto}. Seja $n_\ell \coloneqq |D_\ell|$. Note que existe uma solução para a instância $(F_\ell,D_\ell,c,k^*)$ do problema das $k$-medianas com custo $n_\ell$ em que abrimos todas as instalações que representam subconjuntos que estão em uma mesma resposta ótima da cobertura de conjuntos e associamos os clientes restantes às instalações com custo de conexão 1.
Seja $F_\ell'$ o conjunto das instalações devolvido pela chamada do algoritmo na linha~6 e $D_\ell'$ o conjunto de clientes encontrado na linha~8. Denote $p_\ell \coloneqq  |D_\ell'|/n_ell$. É fácil notar que o custo da solução $F'_\ell$ é
\[ p_\ell n_\ell + 3 (1-p_\ell) n_\ell \leq \alpha  n_\ell\]
e, portanto, 
\[
3 - 2p_\ell \leq \alpha.
\]

Seja $c$ uma constante entre 0 e 1 que vamos escolher depois. Suponha que ${p_\ell \leq 1 - e^{-\frac{1}{c}}}$ para algum $\ell$. 
Então
\[ 1 + 2e^{-\frac{1}{c}} \leq 3 - 2p_\ell < \alpha\]

O valor de $c$ que maximiza o lado esquerdo da desigualdade é $c$ próximo de 1 e teremos $1.735 \leq \alpha < 1.735$ o que é um absurdo. Note que o lado esquerdo da desigualdade é quem nos traz o valor do resultado de inaproximabilidade para o problema das $k$-medianas.

Agora, vamos mostrar que, se $p_\ell > 1 - e^{ - \frac{1}{c}}$ para todo $\ell$, então {\sc Inaprox-JMMSV} devolve uma $(c'\ln n)$-aproximação para o problema da cobertura de conjuntos com $c' < 1$. Vamos chamar de $r$ a quantidade de iterações que o laço {\bf Enquanto} realiza. É evidente que $|I_{k^*}| \leq k^*r$, logo a razão de aproximação desse algoritmo é no máximo $r$. Para todo $\ell$ vale que $p_\ell > 1 - e^{ - \frac{1}{c}}$, então temos que $1 < c \ln\left( \frac{1}{1-p_\ell}\right)$. Note que $n_{\ell + 1} = n_\ell(1-p_\ell)$ e $n_r \geq 1$. Assim, vale que $n_1 \prod_{\ell =1}^{r-1} (1 - p_\ell) = n_r \geq 1$ e, consequentemente, $\ln\left(\prod_{\ell =1 }^{r-1} \frac{1}{1-p_\ell}\right) \leq \ln n_1 = \ln n$. Assim,
\[ r = (r - 1) + 1 < c\sum_{\ell = 1}^{r-1} \ln \left( \frac{1}{1-p\ell}\right) + 1 \leq c \ln n + 1 \leq c' \ln n, \]
em que a última desigualdade vale para algum $c' \in (c,1)$, uma vez que, pelo enunciado do teorema, podemos assumir que $n$ é suficientemente grande.

Assim, temos uma aproximação para o problema de cobertura de conjuntos com razão de aproximação estritamente menor que $\ln n$ para $n$ suficientemente grande.
\end{proof}

Juntando esse resultado com o resultado do Feige temos que a existência de uma $\alpha$-aproximação para o problema das $k$-medianas com $\alpha < 1.735$ implica a existência de um algoritmo determinístico que leva tempo $O(n^{O(\log \log n)})$ para todo problema $\NP$-completo.