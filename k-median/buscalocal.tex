Aqui falaremos sobre custos de transporte entre duas instalações, embora o custo de transporte esteja definido apenas para um cliente e uma instalação.
Podemos definir o custo de transporte entre duas instalações como o menor custo de transporte entre essas duas instalações e um cliente qualquer. Então a desigualdade triangular valerá da seguinte forma: para todo $i,j,k \in F \cup D$ 
\[ c_{ij} \leq c_{ik} + c_{kj}.\]

Um algoritmo de busca local começa com uma solução viável para o problema e checa se alguma alteração local melhora o custo da solução atual. Caso essa melhora ocorra, essa alteração é feita. Esse processo se repete até que não exista alteração que melhore o custo e, então, encontramos uma solução que é chamada de \emph{localmente ótima}. A implementação direta da ideia da busca local na maioria das vezes não garante execução em tempo polinomial e, nesses casos, as alterações só acontecerão se satisfizerem alguma restrição. 

O algoritmo contará apenas com uma operação de troca em que fechamos uma instalação aberta e abrimos uma instalação fechada. Operações que exclusivamente abram ou fechem instalações não são viáveis, uma vez que apenas abrir uma instalação faz com que nossa solução se torne inviável e é fácil notar que somente fechar alguma instalação não melhora o valor da solução. A solução inicial terá  $k$ instalações abertas escolhidas de maneira arbitrária e um cliente sempre estará ligado à instalação aberta mais próxima a ele. Uma operação só será feita se melhorar o custo da solução atual em uma razão de $1-\delta$, para um $\delta>0$. Como essa solução não é necessariamente localmente ótima, iremos chamá-la de solução \emph{quase localmente ótima}. Ao final, mostraremos que, dado um $\eps > 0$, o algoritmo é uma $5(1 + \epsilon)$-aproximação para o problema de $k$-mediana, onde o $\delta$ será escolhido em função do $\epsilon$.

\begin{theorem}
    Para qualquer entrada do $k$-mediana, uma solução quase localmente ótima em respeito a operação de troca tem custo no máximo $5(1+\epsilon)$ vezes o custo de uma solução ótima.
\end{theorem}
\begin{proof}
Seja $S$ o conjunto de instalações que serão abertas na solução quase localmente ótima e $S^*$ o conjunto de instalações que serão abertas na solução ótima. Considere $\sigma: D\rightarrow S$ e $\sigma^* : D \rightarrow S^*$ funções que mapeiam cada cliente à instalação de $S$ e $S^*$ mais próximo a ele, respectivamente. Defina também a funções $\phi : S^* \rightarrow S$ que mapeia cada instalação em $S^*$ à instalação em $S$ mais próxima a ela.

A priori, iremos construir um conjunto de $k$ trocas que serão chamadas de trocas cruciais. Como $S$ é quase localmente ótima, então nenhuma dessas trocas pode melhorar o custo da solução $S$ em uma razão melhor que $ 1- \delta$. Vamos criar uma particionamento $\{Z,O,T\}$ das instalações em $S$ que será fundamental para construirmos as trocas cruciais tal que
\begin{enumerate}[a.]
\item $i$ pertence a $Z$ se $|\{i^* \in S^* : \phi(i^*) = i\}| = 0$;
\item $i$ pertence a $O$ se $|\{i^* \in S^* : \phi(i^*) = i\}| = 1$;
\item $i$ pertence a $T$ se $|\{i^* \in S^* : \phi(i^*) = i\}| > 1$.
\end{enumerate}
Vamos também criar um particionamento $\{O^*,T^*\}$ de $S^*$ em que $i^*$ pertence a $O^*$ se $\phi(i^*) \in O$ e, caso contrário, pertence a $T^*$. Note que $|O| = |O^*|$. É fácil perceber que $|T^*| = |T\cup Z|$. Além disso, como cada instalação de $T$ é imagem de pelo menos duas instalações de $T^*$ em $\phi$, então $|T| \leq \frac{|T^*|}{2}$ e, portanto, $|Z| \geq \frac{|T^*|}{2}$.

Agora, vamos construir as trocas cruciais. Para cada $i^* \in O^*$, criaremos uma troca com o par $(\phi(i^*),i^*)$. Além disso, vamos criar uma coleção com $|T^*|$ trocas, em cada uma delas retiraremos uma localização de $Z$ e colocaremos uma localização de $T^*$. Essas trocas podem ser escolhidas arbitrariamente, contanto que cada localização de $T^*$ apareça uma vez e cada localização de $Z$ apareça no máximo duas vezes.

Considere uma troca crucial onde trocamos a instalação $i \in S$ pela instalação $ i^* \in S^*$. Seja $S' \coloneqq (S \cup \{i^*\})\setminus \{i\}$, ou seja, o conjunto $S$ após realizarmos a operação de troca. Vamos realocar os clientes. Para cada cliente $j$ tal que $\sigma^*(j) = i^*$, vamos associar $j$ à $i^*$. Para cada localização $j$ tal que $\sigma^*(j) \neq i^*$ e $\sigma(j)=i$, vamos associar $j$ à $\phi(\sigma^*(j))$. Note que isso não funciona se $\phi(\sigma^*(j)) = i$, então vamos mostrar que isso não acontece. Suponha por absurdo que $\phi(\sigma^*(j))=i$. Por definição, sabemos que $i$ não pode pertencer a $Z$. Como nas trocas cruciais as instalações que serão fechadas são sempre de $O$ ou $Z$, então $i$ pertence a $O$. Entretanto, existe apenas uma instalação $h$ em $O^*$ tal que $\phi(h) \in O$ que é $i^*$. Isso é um absurdo, uma vez que $\sigma^*(j) \neq i^*$. A mudança no custo gerada por essa troca é 
\[ \sum_{j:\sigma^*(j) = i^*} (c_{\sigma^*(j)j} - c_{\sigma(j)j}) \ + \sum_{\substack{j : \sigma^*(j)\neq i^*,\\ \sigma(j) = i}} (c_{\phi(\sigma^*(j))j} - c_{\sigma(j)j}).\]
Primeiro, vamos encontrar um limitante superior para o segundo somatório. Pela desigualdade triangular, temos que $c_{\phi(\sigma^*(j))j} \leq c_{\phi(\sigma^*(j))\sigma^*(j)}+c_{\sigma^*(j)j}$. Além disso, por definição de $\phi$, vale que $c_{\phi(\sigma^*(j))\sigma^*(j)}\leq c_{\sigma(j)\sigma^*(j)}$. Aplicando a desigualdade triangular novamente, temos $c_{\sigma(j)\sigma^*(j)} \leq c_{\sigma(j)j} + c_{\sigma^*(j)j}$. Juntando essas desigualdades temos que $c_{\phi(\sigma^*(j))j} \leq c_{\sigma(j)j} + 2c_{\sigma^*(j)j}$ e, consequentemente, $c_{\phi(\sigma^*(j))j} - c_{\sigma(j)j} \leq 2c_{\sigma^*(j)j}$. Portanto, vale que
\begin{subequations}
\begin{align*}
    - \delta C &\leq \sum_{j:\sigma^*(j) = i^*} (c_{\sigma^*(j)j} - c_{\sigma(j)j}) \ + \sum_{\substack{j : \sigma^*(j)\neq i^*, \\ \sigma(j) = i}} (c_{\phi(\sigma^*(j))j} - c_{\sigma(j)j}) \\
    &\leq \sum_{j:\sigma^*(j) = i^*} (c_{\sigma^*(j)j} - c_{\sigma(j)j}) \ + 2\sum_{\substack{j : \sigma^*(j)\neq i^*,\\ \sigma(j) = i}} c_{\sigma^*(j)j}
\end{align*}
\end{subequations}
onde $C$ é o custo da solução quase localmente ótima.

Como essa desigualdade vale para toda troca crucial, podemos somá-las. Vamos chamar de $i_\ell$ e $i_\ell^*$ a instalação de $S$ e de $S^*$ referentes a troca $\ell$, respectivamente. Portanto,
\begin{subequations}
\begin{align*}
- k \delta C &\leq \sum_{\ell = 1}^k \left(  \sum_{j : \sigma^* = i_\ell^*} (c_{\sigma^*(j)j} - c_{\sigma(j)j}) \ + 2 \sum_{\substack{ j : \sigma^*(j)\neq i_\ell^*,\\  \sigma(j) = i_\ell}} c_{\sigma^*(j)j}     \right) \\
&= \sum_{j\in D} (c_{\sigma^*(j)j} - c_{\sigma(j)j}) + 2 \sum_{\ell = 1}^k \ \sum_{\substack{ j : \sigma^*(j)\neq i_\ell^*,\\  \sigma(j) = i_\ell}} c_{\sigma^*(j)j} \\
&= C^* - C + 2 \sum_{\ell = 1}^k \ \sum_{\substack{ j : \sigma^*(j)\neq i_\ell^*,\\  \sigma(j) = i_\ell}} c_{\sigma^*(j)j}
\end{align*}
\end{subequations}
em que $C^*$ é o custo de uma solução ótima. Considere o somatório. É evidente que 
\[\sum_{\substack{ j : \sigma^*(j)\neq i_\ell^*,\\  \sigma(j) = i_\ell}} c_{\sigma^*(j)j} \leq \sum_{\substack{j:\sigma(j) = i_\ell}} c_{\sigma^*(j)j}.\]
Sabemos que cada instalação de $S$ aparece em no máximo duas trocas. Então
\[ \sum_{\ell=1}^k \sum_{j: \sigma(j) = i_\ell} c_{\sigma^*(j)j} \leq 2\sum_{i \in S} \sum_{j: \sigma(j) = i} c_{\sigma^*(j)j} = 2 C^*.\]
Assim, conseguimos concluir que 
\[ - k \delta C \leq C^* - C + 2 \sum_{\ell = 1}^k \ \sum_{\substack{ j : \sigma^*(j)\neq i_\ell^*,\\  \sigma(j) = i_\ell}} c_{\sigma^*(j)j} \leq 5C^* - C,\]
consequentemente, 
\[C \leq \frac{5C^*}{1 - k\delta} = (5+\epsilon)C^*\]
em que a igualdade vale se definirmos $\delta \coloneqq \frac{\epsilon}{(5+\epsilon)k}$.
\end{proof}